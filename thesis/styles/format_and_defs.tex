% This is the style file of the BKU report template.
% Copyright (c) 2019/06/20 by Nguyen Van Hung <nvhung1401@gmail.com>.


\documentclass[a4paper, 12pt, twoside, openright]{report}
%-OPTIONS---------------------------------------
% Font size (10pt, 11pt, 12pt)
% Paper size and format (a4paper, letterpaper, etc.)
% Draft mode (draft)
% Multiple columns (onecolumn, twocolumn)
% Formula-specific options (fleqn and leqno)
% Landscape print mode (landscape)
% Single- and double-sided documents (oneside, twoside)
% Titlepage behavior (notitlepage, titlepage)
% Chapter opening page (openright, openany)
%-----------------------------------------------


%%%%%%%%%%%%%%%%%%%%%%%%%%%%%%%%%%%%%%%%%%%%%%%%
% Load basic packages
%%%%%%%%%%%%%%%%%%%%%%%%%%%%%%%%%%%%%%%%%%%%%%%%

% \usepackage[utf8x]{inputenc}
\usepackage[utf8]{inputenc} % The default since 2018
\DeclareUnicodeCharacter{200B}{{\hskip 0pt}}
\usepackage{graphicx}
\usepackage{tikz}
\usetikzlibrary{automata, arrows, calc, arrows.meta}
\usepackage{float}
\usepackage{booktabs}
\usepackage{lscape}
\usepackage{array}
\usepackage{amssymb}
\usepackage{amsmath}
\usepackage{multirow}
\usepackage{longtable}
\usepackage{pifont, chemarrow}
\usepackage{scrextend}
\usepackage{cprotect}
\usepackage{setspace}
\usepackage{titlesec}
\usepackage{fancyvrb}
\usepackage{upquote}
\usepackage[labelfont=bf]{caption}
\usepackage[labelfont=bf]{subcaption}
\usepackage{amsthm,thmtools}

\declaretheorem{theorem}
\declaretheorem[style=definition]{definition}

\usepackage[
	linesnumbered,
	commentsnumbered,
%	ruled,
	boxed,
	resetcount,
	algochapter,
%	noline
]{algorithm2e}
\usepackage[
	breaklinks,
	colorlinks=true,
	linkcolor=black,
	filecolor=magenta,
	citecolor=blue,
	urlcolor=blue,
	unicode
]{hyperref}
\newcommand\fullref[1]{
	\noindent\textbf{\autoref{#1}. \nameref{#1}.}\hspace{5mm}
}

% end of load basic package


%%%%%%%%%%%%%%%%%%%%%%%%%%%%%%%%%%%%%%%%%%%%%%%%
% Load other packages and install template
%%%%%%%%%%%%%%%%%%%%%%%%%%%%%%%%%%%%%%%%%%%%%%%%

% %%% Rename--------------------------------------
% \AtBeginDocument{
% 	\renewcommand{\listfigurename}{Danh mục hình ảnh}
% 	\renewcommand{\figurename}{Hình}
% 	\renewcommand{\figureautorefname}{Hình}
% 	\renewcommand{\listtablename}{Danh mục bảng biểu}
% 	\renewcommand{\tablename}{Bảng}
% 	\renewcommand{\tableautorefname}{Bảng}
% 	\renewcommand{\equationautorefname}{Công~thức}
% 	\renewcommand{\listalgorithmcfname}{Danh sách giải thuật}
% 	\renewcommand{\algorithmcfname}{Mã}
% 	\renewcommand{\algorithmautorefname}{Mã}
% 	\renewcommand{\chapterautorefname}{Chương}
% 	\renewcommand{\appendixautorefname}{Phụ~lục}
% 	\renewcommand{\sectionautorefname}{Mục}
% 	\renewcommand{\subsectionautorefname}{Mục}
% 	\renewcommand{\subsubsectionautorefname}{Mục}
% }

%%% Page margins--------------------------------
\makeatletter
\if@twoside
	\usepackage[top=2.5cm, bottom=2.5cm,left=3cm, right=2cm]{geometry}
	\raggedbottom
\else
	\usepackage[top=2.5cm, bottom=2.5cm,left=3cm, right=2cm]{geometry}
\fi
\makeatother

%%% Header and Footer---------------------------
\usepackage{fancyhdr}
\pagestyle{fancy}
% Header
\fancyhead{}
\setlength{\headheight}{30pt}
\renewcommand{\chaptermark}[1]{\markboth{#1}{}}
\makeatletter
	\if@twoside
		\fancyhead[OL]{\textit{\MakeUppercase\leftmark}}
		\fancyhead[OR]{\thepage}
		\fancyhead[EL]{\thepage}
		\fancyhead[ER]{
			\ifnum\value{chapter}=0
				\textit{\MakeUppercase\leftmark}
			\else
				\textit{\MakeUppercase\chaptername\ \thechapter}
			\fi
		}
	\else
		\fancyhead[L]{
			\ifnum\value{chapter}=0
				\textit{\MakeUppercase\leftmark}
			\else
				\textit{\MakeUppercase\chaptername\ \thechapter\ \MakeUppercase\leftmark}
			\fi
		}
		\fancyhead[R]{\thepage}
	\fi
\makeatother
\renewcommand{\headrulewidth}{0pt}
% Footer
\fancyfoot{}

%%% List of abbreviations-----------------------
%\usepackage[refpage]{nomencl}
\usepackage{nomencl}
\renewcommand{\nomname}{List of abbreviations}
\def\pagedeclaration#1{\dotfill\nobreakspace\hyperpage{#1}}
\makeatletter
\newcommand{\closenomencl}
{%
	\closeout\@nomenclaturefile%
}
\makeatother
\newcommand{\writenomencl}[1]
{%
	\closenomencl%
	\IfFileExists{#1.nlo}{%
		\write18
		{%
			makeindex -s nomencl.ist -o #1.nls -t #1.nlg #1.nlo%
		}% 
	}{\typeout{Nothing there}}%
}
\AtEndDocument{\writenomencl{\jobname}}
\makenomenclature
\renewcommand{\nompreamble}{
	% \textit{List of abbreviations used in this thesis.}
%	\begin{multicols}{2}
}
\renewcommand{\nompostamble}{
%	\end{multicols}
}

%%% Mini TOC------------------------------------
\usepackage{minitoc}
\mtcindent=10pt
\mtcsettitle{minitoc}{}
\mtcsetrules{minitoc}{off}
\mtcsetfeature{minitoc}{after}{\vspace{-30pt}}
\mtcsetfeature{minitoc}{before}{\vspace{-30pt}}
\setcounter{minitocdepth}{1}
\let\oldminitoc\minitoc
\renewcommand{\minitoc}{
	\vfill
	{\large\bfseries Contents}
	\noindent\rule{\textwidth}{0.4pt}
	\adjustmtc
	\oldminitoc
	\noindent\rule{\textwidth}{0.4pt}
	\newpage
}

%%% TOC-----------------------------------------
\usepackage[nottoc, notlot, notlof]{tocbibind}
\usepackage{tocloft, calc}
\setcounter{tocdepth}{1}
\setcounter{secnumdepth}{3}
% Format TOC title
\renewcommand{\cfttoctitlefont}{
	\vspace{-2cm}
	\hfill\Huge\scshape
}
\renewcommand{\cftaftertoctitle}{
	\vspace{3mm}
	\hrule
	\vspace{-5mm}
}
% Format LOF title
\renewcommand{\cftloftitlefont}{
	\vspace{-2cm}
	\hfill\Huge\scshape
}
\renewcommand{\cftafterloftitle}{
	\vspace{3mm}
	\hrule
	\vspace{-5mm}
}
% Format LOT title
\renewcommand{\cftlottitlefont}{
	\vspace{-2cm}
	\hfill\Huge\scshape
}
\renewcommand{\cftafterlottitle}{
	\vspace{3mm}
	\hrule
	\vspace{-5mm}
}
% and uppercase chapter title-------------------
\renewcommand{\cftchappresnum}{Chapter }
\AtBeginDocument{\addtolength\cftchapnumwidth{\widthof{\bfseries Chapter }}}
\makeatletter
	\def\@chapter[#1]#2{
		\ifnum \c@secnumdepth >\m@ne
			\refstepcounter{chapter}%
			\typeout{\@chapapp\space\thechapter.}%
			\addcontentsline{toc}{chapter}{\protect\numberline{\thechapter}\texorpdfstring{\MakeUppercase{#1}}{#1}}%
		\else
			\addcontentsline{toc}{chapter}{\texorpdfstring{\MakeUppercase{#1}}{#1}}%
		\fi
		\chaptermark{#1}%
		\addtocontents{lof}{\protect\addvspace{15\p@}}%
		\addtocontents{lot}{\protect\addvspace{15\p@}}%
		\if@twocolumn
			\@topnewpage[\@makechapterhead{#2}]%
		\else
			\@makechapterhead{#2}%
			\@afterheading
		\fi
	}
	\g@addto@macro\appendix
	{
		\addtocontents{toc}{\protect\renewcommand{\protect\cftchappresnum}{Appendix}}
		\renewcommand{\chaptername}{APPENDIX}
	}
\makeatother
\newcommand{\listofcontents} {
	\dominitoc
	\tableofcontents
	\mtcaddchapter  
	\mtcaddchapter
	\adjustmtc
	\adjustmtc
}

%%% List of algorithms--------------------------
\let\oldlistofalgorithms\listofalgorithms
\renewcommand{\listofalgorithms}
{
	\begingroup
	\let\clearpage\relax
	\oldlistofalgorithms
	\endgroup
}

%%% Chapter page--------------------------------
%\newcommand\capfirstword[1]{\capfirstwordhelper#1\relax}
%\def\capfirstwordhelper#1#2\relax{\MakeUppercase{#1}\MakeLowercase{#2}}
\makeatletter
	\def\@makechapterhead#1{%
		{
			\hfill
			\fontfamily{put}\fontseries{b}\fontsize{95pt}{0pt}\selectfont\thechapter\par
			\vspace{5mm}
		}
		{
			\hfill
			\Huge\scshape#1\par
			\vspace{-10mm}
			\noindent\rule{\textwidth}{0.4pt}\par
		}
	}
	\def\@makeschapterhead#1{%
		{
			\hfill
			\Huge\scshape#1\par
			\vspace{-10mm}
			\noindent\rule{\textwidth}{0.4pt}\par
		}
	}
	\let\ps@plain\ps@empty
\makeatother

%%% Empty page----------------------------------
\newcommand\newemptypage{
	\newpage
	\null
	\thispagestyle{empty}
	\newpage
}

%%% More extra space for new paragraph----------
\setlength{\parskip}{10pt}
\setlength{\parindent}{0cm}

%%% Example environment-------------------------
\newcounter{example}[chapter]
\newcommand{\exampleautorefname}{Example}
\newcommand{\examplename}{Example}
\let\oldtheexample\theexample
\renewcommand{\theexample}{\thechapter.\oldtheexample.}
\newenvironment{example}[1][]
{
	\refstepcounter{example}\par\medskip
	\textbf{\examplename~\theexample #1}
	\rmfamily
}
{
	\medskip
}

%%% List of footnotes---------------------------
% \newlistof[page]{footnotes}{fnt}{List of footnotes}
% %\cftpagenumbersoff{footnotes}
% \let\oldfootnote\footnote
% \renewcommand\footnote[1]
% {
% 	%\refstepcounter{footnotes}
% 	\oldfootnote{#1}
% 	\addcontentsline{fnt}{footnotes}
% 	{
% 		\protect
% 		\numberline{$^{\thefootnote}$}
% 		\hspace{-20pt} #1
% 	}
% }
% \makeatletter
% \@newctr{footnote}[page]
% \makeatother
% \renewcommand{\thefootnote}{\roman{footnote}}
% \deffootnote{3em}{0em}{\thefootnotemark\hspace{3mm}}

%%% Figure environment--------------------------
\let\oldfigure\figure
\let\endoldfigure\endfigure
\renewenvironment{figure}[1][]
{
	\ifx\\#1\\
		\oldfigure
	\else
		\oldfigure[#1]
	\fi
	\let\oldsubref\subref
	\renewcommand\subref[1]{\textbf{(\oldsubref{##1})}}
	\newcommand\sourcefig[1]{(Nguồn: ##1)}
}
{
	\endoldfigure
}

%%% Code environment----------------------------
\usepackage{listings}
\definecolor{commentcolor}{RGB}{168, 86, 0}
\definecolor{keywordcolor}{RGB}{118, 6, 135}
\definecolor{stringcolor}{RGB}{168, 24, 31}
\definecolor{functioncolor}{RGB}{0, 0, 255}
\lstset
{
	language=Python,                            % language programming
	basicstyle=\ttfamily,                       % style for basic
	keywordstyle=\color{keywordcolor}, 			% style for keywords
	identifierstyle=\color{black},              % style for identifier
	stringstyle=\color{stringcolor},            % style for string
	commentstyle=\color{commentcolor},          % style for comment
	morekeywords={as},							% add more keywords
	emph={UNet, self},                          % first list of another word
	emphstyle=\color{functioncolor},           	% style for first list of another word
%	emph=[2]{3.5, sss},                         % second list of another word
%	emphstyle=[2]\color{red},               	% style for second list of
	tabsize=2,                                  % width of tab
%	showtabs=true,                              % enable character tab
%	tab=\rightarrowfill,                        % character tab
%	showspaces=true,                            % enable character spaces in all code
	showstringspaces=false,                     % enable character spaces in string
	breaklines=true,                            % enable breaklines mode
%	escapechar=\#,                              % character to back to LaTeX mode
	numbers=left,                               % lines numbering on left
	numberstyle=\normalsize,                    % stype for lines number
%	firstnumber=100,                            % first number counting for lines number
	stepnumber=1,                               % step lines numbering
	numbersep=10pt,                             % space between lines number and code
%	frame=leftline,                             % insert line for lines numbering
	frame=single,                               % insert box with single line
	framerule=1pt,                              % thickness of line of box
	xleftmargin=20pt,                           % left margin
%	xrightmargin=1cm,                           % right margin
%	backgroundcolor=\color{backgroundcolor},    % white background
	captionpos=b,                               % position of caption
}

%%% Gantt chart---------------------------------
\usepackage{pgfgantt}
\newcommand\Dganttbar[4]{%
	\ganttbar{#1}{#3}{#4}\ganttbar[inline,bar label font=\footnotesize]{#2}{#3}{#4}
}

%%% Hide section, subsection, subsubsection in TOC
%%% but numbered in actual heading--------------
\newcommand{\hidesection}[2][]
{
	\phantomsection
	\ifx\\#1\\
		\section*{\refstepcounter{section}\thesection\quad #2}
	\else
		\section*{\refstepcounter{section}\label{#1}\thesection\quad #2}
	\fi
}
\newcommand{\hidesubsection}[2][]
{
	\phantomsection
	\ifx\\#1\\
		\subsection*{\refstepcounter{subsection}\thesubsection\quad #2}
	\else
		\subsection*{\refstepcounter{subsection}\label{#1}\thesubsection\quad #2}
	\fi
}
\newcommand{\hidesubsubsection}[2][]
{
	\phantomsection
	\ifx\\#1\\
		\subsubsection*{\refstepcounter{subsubsection}\thesubsubsection\quad #2}
	\else
		\subsubsection*{\refstepcounter{subsubsection}\label{#1}\thesubsubsection\quad #2}
	\fi
}

%%% Signature environment-----------------------
\newenvironment{signature}[1]
{	
	\ifnum #1 = 1
		\def\spacewidth{.5\linewidth}
		\def\pagewidth{.5\linewidth}
		\def\nextelement{\par}
	\else
		\ifnum #1 = 2
			\def\spacewidth{0pt}
			\def\pagewidth{.5\linewidth}
			\def\nextelement{}
		\else
			\ifnum #1 = 3
				\def\spacewidth{0pt}
				\def\pagewidth{.333\linewidth}
				\def\nextelement{}
			\fi
		\fi
	\fi
	\newcommand{\addrdate}[4][]
	{
		\noindent\hspace{\spacewidth}
		\begin{minipage}{\pagewidth}
			\centering
			\ifx\\##1\\
				\textit{##2/##3/##4}
			\else
				\textit{##1, ##2/##3/##4}
			\fi
		\end{minipage}
		\nextelement
	}
	\newcommand{\position}[1]
	{
		\noindent\hspace{\spacewidth}
		\begin{minipage}{\pagewidth}
			\centering
			\textbf{\MakeUppercase{##1}}\\
			\textit{\footnotesize(Signature)}
		\end{minipage}
		\nextelement
	}
	\newcommand{\signfig}[2]
	{
		\noindent\hspace{\spacewidth}
		\begin{minipage}{\pagewidth}
			\centering \includegraphics[width = ##1]{##2}
		\end{minipage}
		\nextelement
	}
	\newcommand{\name}[1]
	{
		\noindent\hspace{\spacewidth}
		\begin{minipage}{\pagewidth}
			\centering ##1
		\end{minipage}
		\nextelement
	}
	\vspace{10mm}
}{}

%%% Reference-----------------------------------
\usepackage
[
	backend=biber,
	style=ieee, 
	sorting=none
]{biblatex}
\addbibresource{references/refs.bib}
\AtBeginBibliography{\vspace*{10pt}}
\setlength\bibitemsep{15pt}

%%% Keyword-------------------------------------
\usepackage{imakeidx}
\makeindex
[
	%columns=3, 
	title=List of keywords, 
	options=-s styles/indexstyle.ist
]
% \indexprologue{\textit{List of keywords mentioned in this thesis in alphabetic order.}}
\newcommand\indexpage[1]{\hyperpage{#1}\normalsize}
\let\oldindex\index
\renewcommand\index[1]{\oldindex{#1|indexpage}}

%%% Paragraph-------------------------------------
\titleformat{\paragraph}
{\normalfont\normalsize\bfseries}{\theparagraph}{1em}{}
\titlespacing*{\paragraph}
{0pt}{3.25ex plus 1ex minus .2ex}{1.5ex plus .2ex}
% end of load other packages and install template

