\chapter{Related works}\label{chap: Related works}
\minitoc


\section{Significant works}
Our topic is not new to the Blockchain community.There are tons of paper proposed the same problem as ours. We analyzed the paper Towards on Blockchain Data Privacy Protection (Zexu Wang, Bin Wen and Ziqiang Luo) and find a very interesting approach to the wallet's software architecture. The paper focused on wallet keys protocols, like tree structure of the HD wallet, rules of deriving child keys from given parent keys, system privacy and how to establish a secure collaboration in blockchain network.
We may or may not implement their methods in our thesis since we haven't examine the impact of attacks on their complete system architecture. There are no number prove that their proposed system are perfectly secure.
In practice, BIP32 of Bitcoin wallet is widely used but the methods can not apply to Edwards curve. 
This is a great example of constructing a parent-child keys derivation. We will mention more of BIP32 on our final thesis report.

\section{Other works}
On the other hand, there are official works on HD wallet for Secp256k1 that we would like to include. First of all is the BIP0032 \cite{github/bip0032}, a Bitcoin Improvement Proposal that describes hierarchical deterministic wallets. The specification is intended to set a standard for deterministic wallets that can be interchanged between different clients. Although the wallets described here have many features, not all are required by supporting clients. The specification consists of two parts. In a first part, a system for deriving a tree of key pairs from a single seed is presented. The second part demonstrates how to build a wallet structure on top of such a tree.