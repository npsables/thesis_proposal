\chapter*{Preface}
\thispagestyle{fancy}
\addcontentsline{toc}{chapter}{Preface}
\label{preface}
\hspace*{5cm}
% Preface

In its simplest form, decentralized finance (DeFi) is a system by which financial products become available on a public decentralized blockchain network, making them open to anyone to use, rather than going through middlemen like banks or brokerages. Unlike a bank or brokerage account, a government-issued ID, Social Security number, or proof of address are not necessary to use DeFi. More specifically, DeFi refers to a system by which software written on blockchains makes it possible for buyers, sellers, lenders, and borrowers to interact peer to peer or with a strictly software-based middleman rather than a company or institution facilitating a transaction.

Multiple technologies and protocols are used to achieve the goal of decentralization. For example, a decentralized system can consist of a mix of open-source technologies, blockchain, and proprietary software. Smart contracts that automate agreement terms between buyers and sellers or lenders and borrowers make these financial products possible. Regardless of the technology or platform used, DeFi systems are designed to remove intermediaries between transacting parties.

Though the volume of trading tokens and money locked in smart contracts in its ecosystem has been growing steadily, DeFi is an incipient industry whose infrastructure is still being built out. Regulation and oversight of DeFi are minimal or absent. In the future, however, DeFi is expected to take over and replace the rails of modern finance.
\cleardoublepage