\chapter*{Preface}
\thispagestyle{fancy}
\addcontentsline{toc}{chapter}{Preface}
\label{preface}
\hspace*{5cm}

%%%%%%%%%%%%%%%%%%%%%%%%%%%%% DeFi at first glance %%%%%%%%%%%%%%%%%%%%%%%%%%%%%
% What is DeFi?
We are in the Digital Era, where technologies play a prominent role. We shift from traditional industry to an economy based on information and communication technology. As the fourth industrial revolution has come, decentralized finance has appeared. In its simplest form, decentralized finance (DeFi) is a system by which financial products become available on a public decentralized blockchain network, making them open to anyone to use, rather than going through middlemen like banks or brokerages. Unlike a bank or brokerage account, DeFi doesn't need a government-issued ID, Social Security number, or proof of address. More specifically, DeFi refers to a system by which software written on blockchains makes it possible for buyers, sellers, lenders, and borrowers to interact peer-to-peer or with a strictly software-based middleman rather than a company or institution facilitating a transaction.

Multiple technologies and protocols are used to achieve the goal of decentralization. For example, a decentralized system can consist of a mix of open-source technologies, blockchain, and proprietary software. Smart contracts that automate agreement terms between buyers and sellers or lenders and borrowers make these financial products possible. Regardless of the technology or platform used, DeFi systems are designed to remove intermediaries between transacting parties.

% Blockchain - the core technology of DeFi
Blockchain technology is the heart of any DeFi system. Blockchain differs itself from other databases by it's ability to be transparent, immutable and distributed. Rather than keeping the data censored at a central database, blockchain distributes the uncensored data. That means, anyone can track and see what is happening with their own data. Then comes immutability, which helps DeFi to prevent the negativity of the intermediaries. Now you can access and have control over your data on your own, whereas only the intermediaries can do that in the traditional centralize finance (CeFi) system. Finally, since blockchain is distributed to the network, the DeFi system can tolerate the single-point-failure in comparison with the CeFi system: even if the bank goes down due to network problem or even bankrupt, the user's asset still remain safe and accessible.

% Introduce cryptocurrencies
One of the biggest changes that DeFi brings to modern finance is cryptocurrency. Cryptocurrency and fiat currency have a bit of common ground in that neither of them is backed by a physical commodity - but that’s where the similarity ends. While fiat money is controlled by governments and central banks, cryptocurrencies are essentially decentralized. As a digital form of money, cryptocurrencies have no physical counterpart and are borderless, making them less restrictive for worldwide transactions.

% Sum up
Though the volume of trading tokens and money locked in smart contracts in its ecosystem has been growing steadily, DeFi is an incipient industry whose infrastructure is still being built out. Regulation and oversight of DeFi are minimal or absent. In the future, however, DeFi is expected to take over and replace the rails of modern finance.

\cleardoublepage