\chapter*{Preface}
\thispagestyle{fancy}
\addcontentsline{toc}{chapter}{Preface}
\label{preface}
\hspace*{5cm}
% Preface

There are two approaches to defense a website in this field: active defense and passive defense. With the expansion of data, it is harder to manually identify and counter cyber threats. However, the rise of using machine learning-based approaches to counter cybersecurity is emerging, with vast applications of machine learning in cybersecurity providers.

In this thesis, we are proposing two machine learning-based approaches to active defense and passive defense. To protect websites actively, we are trying to counter the most common damages to websites: phishing. We have developed a machine learning-based module to actively search for potential phishing websites, verify it and raise alarms about phishing threats online. To the passive defense, traditionally, rule-based WAFs are commonly used. However, they have a high false-positive rate. We are building a machine learning-based validator to validate the requests to support WAFs.

Technically, the phishing website detector will generate a suspicious domain from the legitimate website domain. It will check whether the suspicious website is a phishing one, by comparing the screenshots and textual contents of the suspicious and original one, using a specialized screenshot similarity extraction and GloVe embedding respectively. Then the similarities will be fed to a machine learning model to detect the phishing website.

On the other hand, the malicious request validator is based an observation that legitimate requests to a website usually belong to the same category. The module uses a Convolutional Neural Network to category the suspicious request and checks if that request is in the same category as the normal requests observed or not. This result and the classification of WAF is combined into the final decision.

The dataset for phishing detection is collected from Phishtank. The dataset includes 100 phishing websites and their 64 original websites, forming a dataset with 16300 pairs of phishing-original websites. For malicious request validator, we crawl code snippets from GitHub, as most request payloads are structured languages. We have 127686 records of data in three categories: plain text, JavaScript, and PHP. Also, CSIC 2020 end ECML/PKDD 2007 datasets have been used for verification.

The best experimental results for phishing website detector is a Recall of $96\%$. The request classifiers achieved almost zero false positive rate and average precision of $95.86\%$. The malicious request validating module has achieved $96\%$ True Positive Rate and $37\%$ detection rate for CSIC 2020 dataset, while the figure for ECML/PKDD 2007 is $91\%$ and $51\%$. The module, though not yet completed and cover all the cases to be applicable, show good results in detection XSS attacks and SSI attacks, in which involve the use of JavaScript and PHP in the request classifier.
\cleardoublepage