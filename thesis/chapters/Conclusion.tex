\chapter{Conclusion and Future Work} \label{chap:conclusion}

% \section{conclusion}
% The field diagram detection is an interesting and complex study domain. Although there are many research about on-line diagram, there is still less research about recognizing off-line diagram. With this thesis proposal, we have achieved new background knowledge about the computer vision. We are able to access each procedures in image processing which is also a new field for us. We finds many approaches for handwriting flowchart recognition and we select the suitable approaches using Faster R-CNN, YOLO v4 and RefineDet.\\
% During the research process, our group still had many problems in teamwork, lack of time management skills which affects negatively our result. This problems may become a wall in our work in the future, which we need to overcome and improve our working process.

% \section{Challenges}
% \begin{itemize}
%     \item One of the biggest problem with server-client system is scaling or the amount of customers serving at the same time. Most of our experiment is conducted in the local machine so that there is a need to find a way to provide more stable service in the final product.
%     \item The target is to support a wide range of products so as many devices can install this app as possible. Then it can lead to performance issues and the need to find a balance between stable functioning and feature variety.
%     \item Security is also important since many of these data can be very crucial. All of the information sent or receive by the app and the server need to be secured in order to prevent data leak.
% \end{itemize}

% \section{Future work}
% In the future, we are going to implement the flowchart recognition system and also evaluate the capabilities of the programming languages and the support framework to reduce the constraints of our proposed system while solving all of the listed challenges.