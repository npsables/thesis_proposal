\chapter{Introduction} \label{Introduction}
\minitoc

\section{Overview}

\subsection{Problem statements}

The use of technology in financial services is not new. Most transactions at banks or other financial services companies are accomplished with the help of technology nowadays. However, the role of technology is restricted to being a facilitator of such transactions. Companies still have to contend with navigating the legalese of jurisdictions, competing financial markets, and different standards to make a transaction possible. With its stack of common software protocols and public blockchains to build them on, DeFi places technology at the front and center of transactions in the financial services industry.

DeFi services and apps are mostly built on public blockchains, and they either replicate existing offerings built on the rails of common technology standards or they offer innovative services custom-designed for the DeFi ecosystem. At the same time, DeFi applications provide users with more control over their money through personal wallets and trading services that explicitly cater to individual users instead of institutions.

Cryptocurrencies are the “money” of the DeFi ecosystem. A cryptocurrency is a digital or virtual currency that is secured by cryptography, which makes it nearly impossible to counterfeit or double-spend. Many cryptocurrencies are decentralized networks based on blockchain technology – a distributed ledger enforced by a disparate network of computers. A defining feature of cryptocurrencies is that they are generally not issued by any central authority, rendering them theoretically immune to government interference or manipulation. They hold the promise of making it easier to transfer funds directly between two parties, without the need for a trusted third party like a bank or credit card company. These transfers are instead secured by the use of public keys and private keys and different forms of incentive systems, like Proof of Work or Proof of Stake.

In modern cryptocurrency systems, a user's "wallet" or account address, has a public key, while the private key is known only to the owner and is used to sign transactions. Fund transfers are completed with minimal processing fees, allowing users to avoid the steep fees charged by banks and financial institutions for wire transfers.

% \subsubsection{HDWallet architect}

% \subsubsection{Protocols}

% \subsubsection{Algorithm}

% \subsection{Explain why this thesis is chosen}

\section{Objectives}

\subsection{Aims}

\subsection{Practical benefits/application}

\section{Scope of the study}

\section{Tentative structure of the study}

\section{Tentative schedule}