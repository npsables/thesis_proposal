\chapter{Introduction} 
\label{Introduction}

\textit{In this chapter, we are going to discuss the overview of the wallet we are creating. Thus, we are going to present the objectives, scope, and structure of this thesis.}
\adjustmtc
\minitoc
\section{Overview}
\label{overview}
\bigskip

At the moment, blockchain is a robust trend in financial technology. Blockchain is atechnology that enables truthless digital currencies transaction without intermediaries. It helps to prevent the double-spending and solve the censorship problem in traditional finance. Similar to a transaction in a bank, you need to sign it to provide the proof of identity. In the blockchain system, you use a digital signatures schema. The private key is kept private and used to sign the transaction, while the public key is accessible by everyone, used to verify the signature. The sender can also use your public key as a public address to send cryptocurrency.

The pair of keys is preserved and managed by using a cryptocurrency wallet, or crypto wallet for short. Some people may call it a digital wallet, but that is not entirely correct. A digital wallet (e-wallet) such as Momo and Paypal can hold users’s digital assets (government-issued currency) and link to their credit/debit cards. The banks apply the online banking service and allow the digital wallet to connect to users’s accounts, which means users are required to sign up the service first to use a digital wallet and they have to transfer indirectly through the banking system. On the other hand, crypto wallet has access to users’s crypto assets, and can help them to transfer cryptocurrencies directly because the blockchain system does not required a trusted third party nor any intermediary. All the transfer process takes in terms of minutes, no matter how far it is from you to the receiver. With some lightning-fast consensus like the Proof of History blockchain, it took less than 10 seconds. In the case of digital wallets or bank accounts, you have to wait for days before your transaction is confirmed, and it even worse when you decide to transfer to another country/state or cross-bank. The bank does have a solution for this problem (NAPAS, Visa, MasterCard,...) but you have to pay an extra fee for the service no matter how much you spend. That is why rather than the traditional stock markets, people are more likely to invest in the decentralized system since they have to stay home due to the COVID pandemic. The cryptocurrency still has some more advantages to traditional finance, but for the scope of our thesis, we won’t include them here. Another difference is that, crypto wallet does not hold any of users’s private information such as phone number, name, ID card, etc (also know as KYC). Therefore; users’s information is protected and will not be leaked even if the crypto wallet got hacked.

In practical, the investors do not invest in a single market. They love to split their funds and put their money into multiple projects because one of the famous investment rule is ``Don’t put all your eggs in one basket". However; a crypto wallet creates a different key pair to sign for each transaction and each crypto wallet is connect to only one platform. The users have to create multiple wallet for each market, and they have to keep track of the used key pairs for both backing up and avoiding key leakage. This is inconvenience for both advanced users and investors since hundreds of transactions are made and hundreds of markets launch per day.

Also, crypto wallet or digital wallet, they are still online applications that rely on the Internet. An advanced user acknowledges that the blockchain system and the internet is not a safe place. Both of the wallets are vulnerable to common attacks such as man-in-the-middle attack, phishing, social engineering, and more. These attacks are hard to prevent because they focus on the human factor: the users’s awareness. In addition, blockchain system is a collection of digital signature, which reckons the digital signature algorithm (DSA). If users lost their private key, or the cryptography behind the DSA is broken, all of their assets are gone and there are no way to retrieve those assets because blockchain system holds no personal information about the users.

\section{Objectives}
\label{objectives}

To solve the first problem, we choose to work on an hierarchical deterministic wallet (HD wallet). The HD secret key derivation and transfer protocol allow creating a huge amount of child keys from parent keys in a hierarchy. Crypto wallets using the HD protocol are called HD wallets. HD wallet helps the users to manage multiple child wallets and backup easier. If a child wallet is lost, the other child wallets remain safety when the parent keys are not leaked.

For the second problem, the main reason why we brought it up is that we found the standard digital signature algorithm (ECDSA) used in the blockchain system has a flaws since 2009 \cite{Schmidt2009} so the need for a higher-security signature is vital. What catches our eyes is the Ed25519 digital signature schema. The release of curve Curve25519 \cite{Bernstein2006} from the SafeCurves project \cite{safecurves}, which has an incredible speed and is applicable in the digital signature Ed25519 \cite{Bernstein2011}, marks a new trend in elliptic curve cryptography (ECC). Also, the expiration of Schnorr’s signature patent in 2008 allows Ed25519, a digital signature based on Schnorr’s signature and the twisted Edwards curve, to be publicly used. The advantages of Ed25519 are comprised in the RFC 8032 document \cite{Josefsson2017}:

\begin{enumerate}
    \item High performance in various platforms.
    \item Generating a random number for each signature is not needed.
    \item Resilience against side-channel attacks.
    \item Complete formulas (e.g. addition law): the formula works for all points in the curve.
    \item Hash collision resistance.
    \item 128-bit security level.
    \item Small 32 byte keys and 64 byte signatures
\end{enumerate}
This is the motivation for our thesis, to develop an HD wallet for Ed25519. We will explain the benefits of the HD wallet, along with implementing and analyzing the Ed25519 of HD wallet’s usage. 

\section{Scope of the study}

In this thesis, our main focus is to study elliptic curve cryptography and its digital signature algorithm to point out the difference between the standard elliptic curve digital signature algorithm (ECDSA) and the Ed25519 signature schema in a cryptanalytical way. For blockchain, we take a look at how to make a transaction using crypto wallet and how a crypto wallet is connected to a blockchain system. Finally, we try to build a prototype for an HD wallet using Ed25519 signature schema, limited to a web application, with an open-source library and the application will support at most 3 blockchains. We look for open-source protocols and choose Solana blockchain as our testing environment because Solana uses Ed25519 for their signatures.

\section{Thesis structure}

The contents of our thesis are presented in seven chapters, namely:

\textbf{Chapter 2} introduces and discusses the background knowledge of this thesis, including the necessary knowledge of the HD wallet and blockchain network, group and field for elliptic curve cryptography, schemas of Ed25519 and its performance as well as the important cryptography functions.

\textbf{Chapter 3} discusses the protocols of the HD wallet, the related works on implemented library and some open source wallet that used in the industry. 

\textbf{Chapter 4} shows our system design for the HD Wallet for Ed25519, how did we solve the challenge on key derivation schema. We also improve the wallet by letting it hold tokens of different blockchains (expanding with Secp256k1 curve).

\textbf{Chapter 5} describes how we actually implement the HD wallet, including the discussion on library that we used.

\textbf{Chapter 6} summarizes the testing results of our application.

\textbf{Chapter 7} concludes our work. Also, we analyze the drawbacks and how to prevent it with the spaces for improvements to the thesis.
