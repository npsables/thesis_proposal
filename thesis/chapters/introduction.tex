\chapter{Introduction} \label{introduction}
\section{Overview}
Research in image recognition has a long journey. Within the image area, one smaller section deals with the detection of diagrams such as flowcharts, UML,... Since the introduction of the Online Handwritten Flowchart Dataset (OHFCD) in 2011, there has been numerous attempts in flowchart recognition. Flowchart recognition consists of two main approaches: online and offline recognition. In online recognition, the diagram is drawn as a sequence of strokes using a device with a touchscreen such as tablet and a pen or finger. In offline recognition, the diagram is a raw image from a source of image capture like phone camera. A few years ago, there has been more attention in online recognition as it reaches higher accuracy than offline recognition. However, when it comes to a pre-drawn diagram, where individual strokes cannot be captured, offline flowchart is preferred, though online recognition can still be constructed. For recognizing objects within images, object detectors based on Convolutional Neural Networks (CNN) are very common. While they can be applied to detect the individual symbols of a flowchart, an off-the-shelf object detector cannot be used to detect the relationships between elements of a graphic.

In captured image, there are many features that negatively affect the recognition process. Some other features are not needed in the process and may cause more problems when implementing the recognition algorithm, for example, color feature of the image is not a necessary feature in flowchart recognition which should be removed to reduce the work in the post-processing. Additionally, image distortion is the most pleasant problems in image processing. Image is distorted due to various type of noise such as Gaussian noise, Speckle noise, Salt and Pepper noise and many more other type of noise \cite{11}. Noise is always appears in digital images during image acquisition, coding, transmission, and processing steps. It may arise in the image as effects of basic physics-like photon nature of light or thermal energy of heat inside the image sensors \cite{10}. Noise tells unwanted information in digital image. Noise produces undesirable effects such as artifacts, unrealistic edges, unseen lines, corners, blurred objects and disturbs background scenes. Therefore, to achieve a good recognition result, pre-processing also plays an important role in the recognition process. Image pre-processing target is an improvement of the image quality that suppresses undesired distortions or enhances some image features relevant for further processing and analysis task. This process includes restoration state, whose aim is reducing or removed noise, negative impacts from environment condition in the image, and enhancement stage, whose goals are enhancing the main features of the image which make the recognition process work more effectively. In addition, some features of the image, which may be unnecessary according to the target of the recognition result, can be removed in pre-processing to reduce the complexity of the recognition algorithm. There are many well-known techniques used in image processing such as grayscale image for color reduction, histogram equalization for image contrast adjustment, noise filter techniques such as Gaussian Blur, Median Blur. In addition, image is able to be converted into binary image, which only includes two colors black and white. There are also the techniques called morphological transformation which is normally used on binary image and can be used to enhance image features. However, these techniques has their own advantages and disadvantages, which needs to be carefully selected to give a acceptable image quality for the recognition process.

This project will not stop with the theory, but also apply it to build a complete product that serves actual clients to solve real-life problems. Nowadays, many meetings require to have a black board so people can express their ideas on. However, there is still a limited option when it takes to saving these sketches and transform them into digital form for storing and referencing in the future. Realizing the need of a product that saving these drawing, we decided to build an application that can convert hand-drawn ideas into digital form. This app should also have login system so that users can save their own data safely in the database and sharing them with their co-worker.

The report is organized as follows Chapter \ref{survey} briefly surveys application that can detect object and related work in diagram detection in general and flowchart detection in particular. Chapter \ref{background} provides sufficient knowledge in order to implement the project. Chapter \ref{chap:ProposedSystem} shows our proposed system, including how the application works. Chapter \ref{chap:Systemdesign} shows the implementation of the application and server. Chapter \ref{chap:exp} lists our experiment and implementation of the system. Finally, Chapter \ref{chap:conclusion} shows our challenge and potential future in the thesis. \\

\section{Project Goals}

The target of this project is to build a system that is able to convert hand drawn flowchart image into digital version and allows user to modify the result into the final product before sharing or converting it into other form.