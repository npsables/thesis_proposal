\chapter{Challenges} \label{chap:Challenges}
\minitoc


\section{Challenge}
There are already a HD wallet for Secp256k1. Fortunately, it's open source and the explanation is clear. We can study how to generate a child key on Secp256k1, then try to apply or adjust the CKD function on Ed25519. The reason why is that both Secp256k1 and Curve25519 (the curve 'used' in Ed25519) share the same form, Montgomery. The differences are:

\begin{itemize}
    \item The curves are defined over distinct fields
    \item Curve25519 is birationally equivalent to Ed25519, not the main curve
    \item Ed25519 is a twisted Edwards curve, while Secp256k1 and Curve25519 are \\ Montgomery curve
\end{itemize}

With that being said, some works on Ed25519 has been done. However; none of them are official. Even NIST is still proposing Ed25519 on draft. That means we have to prove the correctness of the math behind CKD function. This is the most challenging problem. Our plan now is learn about modern cryptography, especially elliptic curve, twisted curve and twisted Edward curve; then we analyze the CKD function of Secp256k1; finally, applied what we have studied to work out the CKD function for Ed25519 with the proof of correctness and implement it into a wallet.

\section{Proposed solutions}
correctness of the math behind CKD function. This is the most challenging problem. Our plan now is learn about modern cryptography, especially elliptic curve, twisted curve and twisted Edward curve; then we analyze the CKD function of Secp256k1; finally, applied what we have studied to work out the CKD function for Ed25519 with the proof of correctness and implement it into a wallet.