\chapter{System implementation} \label{chap:System_implementation}
\textit{In this chapter, we talk about what technology to use and the chosen system solution for our wallet. Since security play a majority part of our thesis, we will examine every technologies that used in our final thesis report.}
\minitoc

\section{Technology used}
{\textit {\textbf{Programing language and libraries}}}
\begin{itemize}
\item \emph{ReactJS} [ref react] for User Interface components. 
\item \emph{Node.js} [ref node] for Server components.
\item \emph{hdkey} and \emph{bip-0032} for studying hierarchical deterministic wallet on Secp256k1
\item \emph{TweetNaCl.js} a crypto library of TweetNaCl in JavaScript
\item \emph{Python} [ref Python] for automation testing.
\end{itemize}

{\textit {\textbf{Blockchain}}}
\begin{itemize}
    \item \emph{Solana} [ref Solana] TestNet and, if we can make it in time, DevNet
\end{itemize}

\section{System design}
{\textit {\textbf{Wallet type}}}
\begin{itemize}
    \item We decided to build a \emph{Hybrid Wallet}. The private key will be hold by only the user, safely saved in the user hardware (phone or computer). The user has full control of their tokens or coins but they have to manage it on their own through the software. Outside of the application, user must keep their private key safe.
    \item From the financial aspect, the wallet is a non-custody wallet. It doesn't store any user information in comparison to a traditional bank's account (KYC). Since a blockchain wallet interacts with a decentralize system, where everything is transparent, it should not use anything related to a user's personal information.
\end{itemize}

{\textit {\textbf{Signature Schema}}}
\begin{itemize}
    \item EdDSA (curve Ed25519) and a proposed CKD function on Ed25519 (our work) for child key generation.
\end{itemize}

{\textit {\textbf{Main function of the wallet}}}
\begin{itemize}
    \item Key generation.
    \item Key management.
    \item Asset management.
    \item Transfer management.
\end{itemize}