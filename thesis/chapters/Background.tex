\chapter{Background} \label{background}

\label{chap:background}
	\textit{In this chapter, we introduce the foundation knowledge of the thesis, including the history and definition of Blockchain Technology, Cryptocurrency, 
	Hierarchical Deterministic Wallet (HD Wallet) and Cryptography}
\minitoc

\section{Blockchain Technology}

\subsection{History and Definition}

The definition of blockchain was introduced to the world by a person (or a group of people) under the name Satoshi Nakamoto on October 31, 2008. 
It was applied to enable the emergence of a "purely peer-to-peer (no financial institution or third party) electronic cash" named Bitcoin where transactions take place in a distributed system.
Infact, Satoshi did not invent blockchain, and Bitcoin blockchain is not the first chain that ever created. 
Back in 1991, cryptographers Stuart Haber and Scott Stornetta published a whitepaper “How to Time-Stamp a Digital Document” in the Journal of Cryptography. 
Their goal is to digital time-stamping of documents so that it is infeasible for a user either to back-date or to forward-date digital document, even with the collusion of a time-stamping service. 
The technology is called a blockchain because the distributed electronic ledger stores items of data in time-stamped digital groups called blocks. Each block includes an alphanumeric code called a “hash” summing up its data. The hash of each completed block also appears in the next one in the chain, which means that to alter one block you would have to alter all the ones connected to it. These cryptographic dominos function together to protect against tampering or fraud.
Base on this theory, the longest running blockchain, also by Haber and Stornetta, started in 1995, publishes the weekly summary hash value every week in the New York Times and still running strong today. 

PICTURE

According to NIST:

Blockchains are distributed digital ledgers of cryptographically signed transactions that are grouped into blocks. Each block is cryptographically linked to the previous one after validation and undergoing a consensus decision. As new blocks are added, older blocks become more difficult to modify. New blocks are replicated across all copies of the ledger within the network, and any conflicts are resolved automatically using established rules. 



\subsection{Bitcoin}



\section{HDWallet}

\subsection{Category}

\subsection{Coins}

\subsection{Wallet structure}

\section{Cryptography}

\subsection{Cryptographic hash}

\subsection{Diffie-Hellman algorithm}

\subsection{RSA}

\subsection{ECC}

\subsubsection{Blockchain: secp256k1}

\subsubsection{Discrete logarithm problem}

\subsection{Twisted-Edward curve and Ed25519}

\subsection{Key derivation function}
