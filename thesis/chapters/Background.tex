\chapter{Background}
\label{chap:background}

\textit{In this chapter, we introduce the foundation knowledge of the thesis, including the Blockchain Technology, Cryptography, and Hierarchical Deterministic Wallet (HD wallet)}

\minitoc

\section{Blockchain Technology}
\label{blockchain}
\bigskip
\subsection{What is blockchain?}

Blockchains are immutable digital ledger systems implemented in a distributed fashion (i.e., without a central repository) and usually without a central authority. The definition of blockchain was introduced to the world by a person (or a group of people) under the name Satoshi Nakamoto on October 31, 2008. It was applied to enable the emergence of a "purely peer-to-peer (no financial institution or third party) electronic cash" named Bitcoin where transactions take place in a distributed system. Satoshi did not invent the blockchain, and the Bitcoin blockchain is not the first chain that was ever created.

The word "blockchain" or "block" and "chain" wasn't used back then. Only when it was known in Satoshi Nakamoto's Bitcoin paper did the term "chain" of "blocks” become a representation for this technology. Later on, the community used that word for Nakamoto's invention. Bound to the emergence of Bitcoin and cryptocurrency, a concise description of blockchain technology is provided by NIST: Blockchains are distributed digital ledgers of cryptographically signed transactions that are grouped into blocks. Each block is cryptographically linked to the previous one (making it tamper evident) after validation and undergoing a consensus decision. As new blocks are added, older blocks become more difficult to modify (creating tamper resistance). New blocks are replicated across copies of the ledger within the network, and any conflicts are resolved automatically using established rules.

\subsection{How does it works?}


To make a blockchain work, it needs a blockchain network formed by many master nodes, a consensus protocol, and the distributed and immutable blockchain ledger. Every master node has a copy of the ledger, and most of the master nodes contribute to “validate and create” the next block to be added to the current chain to create a new ledger. By validation, that means the master node acts as a validator and confirm if a transaction is valid: is the owner of the transaction authorized, is this transaction not a double-spend one, or does this transaction follow the rules, etc. By creation, the master node each validate and put multiples valid transactions into a block, then thanks to the consensus protocols to choose whose block will be added to the chain to form a new ledger. The consensus protocol is an agreement protocol, that helps many master nodes to make a general approval on which blockchain ledger is used for the whole network and dispose of the malicious one.

Suppose that Alice wants to send 10 BTC to Bob, she uses her blockchain wallet to make a transaction, sign it, and then send the signed transaction to the blockchain network. The transaction then waits to be validated and put in a block by the master node. Now inside the blockchain network, the transaction is broadcasted to most of the master nodes to queue. Then each node can choose a number of transactions that are going to be validated and put into a block, which may contains Alice's transaction. After that, the blockchain network works on agreement which is the appropriate block to be chained with the current blockchain using consensus protocol. Bitcoin's consensus is Proof-Of-Work, in which the master node competes against each other by solving a cryptographic hash puzzle to gain the right to add their node to the chain. The puzzle must be too hard to break but fast to verify, so that the network can verify one's work and approve the new block. The first node to find out the nonce will 'mine' the block, put the validated transactions into a block with an extra transaction called coin-base transaction. The coin-base transaction is the reward for the master node, and works as incentives to encourage the node to continue to contribute to mine new blocks faithfully. That's why the term miner is born and being more common to end-users rather than master node. Once a block is mined, the node will broadcast the blockchain appended with the new block to the blockchain network. If that block contains Alice's transaction, her transaction is finished and Bob now has 10 BTC in his blockchain wallet.

\section{HD wallet}
\label{HD wallet}

\subsection{What is a blockchain wallet?}

A blockchain wallet, sometimes referred to as a cryptocurrency wallet or crypto wallet, is a program that allows you to "store", send and receive digital currencies. Since cryptocurrency doesn’t exist in any physical form, your wallet doesn't actually hold any of your coins. Instead, it tracks the transactions you made, which are stored in blockchain, and then infers your balance. Thereby, blockchain is an essential component of a blockchain wallet.

Instead of holding physical coins, a wallet has a public key and a private key. Public key is a long sequence of letters and numbers that can form the wallet address. With this, other people can send money to your wallets. It’s similar to a bank account number in which it can only be used to send money to an account. Private key is used to access the funds stored in the wallet. With this, you can control the funds tied to your wallet’s address. It works like your PIN number, you should keep it 100\% secret and secure. However, it’s worth noting that not all wallets give you sole ownership of your private key, which essentially means that you don’t have full control over your coins. As well as storing your public and private keys, crypto wallets interface with the blockchains of various currencies so that you can check your balance and send and receive funds.

\subsection{Category}

Now that you know what it is, let’s take a closer look at the five different types of wallets available, each with its own advantages and disadvantages in terms of security, ease of use, convenience and a range of other factors. The most common type of wallet out there, desktop wallets are downloaded and installed on your computer. Easy to set up and maintain, most are available for Windows, Linux and Mac, although some may be limited to a particular operating system. Many cryptocurrencies offer a desktop wallet specifically designed for their coin.

Desktop wallets provide a relatively high level of security since they’re only accessible from the machine on which they’re installed. The biggest disadvantage is that they also rely on you to keep your computer secure and free of malware, so antivirus and anti-malware software, a strong firewall and a common-sense approach to security are required to keep your coins safe. Most desktop wallets will provide you with a long string of words upon installation. These words are known as your recovery seed or sentence and map with your private key, so it’s important to store them somewhere safe in case your computer dies or you need to format the operating system and re-install your desktop wallet. Some popular desktop wallets: Electrum, Exodus, Copay.

Mobile wallets are fairly similar to desktop wallets, with the obvious difference being that they run as an app on your smartphone. Mobile wallets feature many of the same advantages and disadvantages as desktop wallets, with your private key stored on your device. Smartphone wallets are often easier to use compared to their desktop counterparts and include the ability to scan other wallet addresses for faster transactions. They also make it simpler to access your coins on the go and use cryptocurrency as part of everyday life. You will need to be extra careful about losing your smartphone, though, because there’s a risk that anyone who has access to your device might also have access to your funds. Choosing an app that allows you to back up your wallet with a 12- or 24-word passphrase is a good idea. Popular mobile wallets: Jaxx, Coinomi, Edge.

Online wallets (most often provided by exchanges but sometimes offered by third parties) are connected to the Internet and are generally the easiest to set up and use. Most only require an email address and a password to create an account, and web wallets are usually designed to provide a simple and straightforward user experience. The biggest advantages to online wallets are that they can’t be lost and that they’re accessible from any computer with an Internet connection. However, being online is unfortunately also their biggest disadvantage. Because some platforms maintain the wallets of thousands of users, they can become hot targets for hackers. It’s also important to check whether the wallet you choose lets you retain complete control of your private keys or whether they’re owned by the wallet provider. Popular web wallets: blockchain.info, MyEtherWallet, Coinbase.

Hardware wallets add another layer of security by keeping your private key on a USB stick or specially designed piece of hardware. They allow the user to plug the USB stick into any computer, log in, transact and unplug – so while transactions are carried out online, your private key is stored offline and protected against the risk of hacking. As a result, hardware wallets are widely considered to offer the most secure storage option. The biggest disadvantage of hardware wallets is that they’ll cost you. Prices vary depending on the model you choose, but they generally cost upwards of \$150. You also need to keep the device safe, but if you do lose your hardware wallet, the device itself is PIN-protected and there are usually other protective measures in place to help you recover your funds. Popular hardware wallets: Ledger Nano X, Ledger Nano S, TREZOR, KeepKey.

Paper wallets take the concept of entirely offline keys used for hardware wallets to the next logical step: simply print out your public and private keys and use that piece of paper as your wallet. As secure as they are, paper wallets are also complex and quite confusing for beginners. They’re typically only used by advanced users who want a high level of security. To transfer money to a paper wallet, you use a software wallet (any of the above mentioned) to send money to the public key printed on the sheet of paper. Most often, this is printed as a QR code for easy scanning. To transfer money from the paper wallet to someone else, you would first need to transfer money to a software wallet (by manually entering the private key into the software), and then transfer money from the software wallet to the recipient as usual. Popular paper wallets: Bitaddress.org, WalletGenerator.net.

As you’re researching and comparing a range of wallets, you’ll probably come across the terms “hot wallet” and “cold wallet”, or perhaps the concept of “cold storage”. So, what does temperature have to do with crypto storage? A wallet is hot when it's connected to the Internet. Nothing on the Internet is 100\% secure, so funds kept in a hot wallet are always at a slight risk of theft or loss from software bugs or hackers. A wallet is cold when it's safely offline and can't be deliberately or accidentally compromised over the Internet.

From an economic perspective, the wallet can be custodial (requires KYC) or non-custodial (doesn’t require KYC). Custody means that your wallet is controlled by a third party, and they need your personal information to authenticate you whenever you want to use your wallet. An example is the Binance “wallet”, or actually, the application provided by Binance. Your private keys are managed by them, which resembles a bank managing your account. Also, you have to provide your ID number, phone number, and email so that every time you want to make a transaction, the application can authenticate the right owner of the wallet. With that, you put faith in a third party to secure your funds and return them if you want to trade or send them somewhere else. While a custodial wallet lessens your responsibility, it requires you to trust in the custodian that holds your funds, which is usually a cryptocurrency exchange. On the other hand, non-custodial crypto wallets give you complete control of your keys and therefore your crypto assets. While some people store large amounts of crypto on exchange accounts, many feel more comfortable with a non-custodial wallet, which eliminates a third-party between you and your crypto. Apart from that, based on an engineering prespective, we cared about how the key is managed. We re-categorize the wallet into three main groups: hardware, software and hybrid. Hardware wallets are only connected to the internet when you need to make a transaction and you have no information about the secret keys, whereas software wallets are connected to the internet and store your secret keys. Hybrid is the combination of both hardware and software wallet, that your secret keys are stored in a secured space and are managed by the software. We will develop on a hybrid wallet as we need some information about the secret keys and need to manage the keys.

\subsection{What is a HD wallet?}

A hierarchical deterministic wallet allows you to generate and manage multiple child wallets. The HD wallet architecture consists of a pair of master private key and master public key and the management software. The software has 3 functionalities: manage the keys, check assets and transfer/deposit assets. For the keys management, the software can store and derive the child key using the CKD function to create a corresponding sub wallet for the native asset of a blockchain. By that way, we can hold multiple types of cryptocurrencies and perform cross-chain actions.

Why an HD wallet? Essentially, the HD wallet helps the user to manage multiple key pairs efficiently. As mentioned above, it is not practical for the wallet to keep track of all generated key pairs, for which each of them is used to sign a transaction. The more transactions the users make, the more key pairs they have to keep track. Instead of backing up every generated key pair, now you only need to backup the master key pair in case of HD wallet. If one of the key pairs is leaked, your wallet and crypto assets are gone. Secondly,  HD wallet offers stronger privacy than non-HD wallet because key pairs are derived automatically for every transaction. Because Bitcoin and other blockchains are public ledgers, address balances are public knowledge. However, as you have multiple addresses, others may not know which addresses are linked to you and which aren’t - provided that you haven’t shared your extended public key. Your extended public key can show all of your balances, and therefore should never be shared. Beyond privacy, HD wallet offers enhanced security because every transaction received is on a different address. Someone would need multiple private keys to access your wallet’s multiple crypto balances. So as long as you haven’t shared your extended private key, your funds should be secure. And most HD wallets make it difficult, if not impossible, to share confidential information such as your master key pair.

\section{Cryptography}
\label{cryptography}

\subsection{Elliptic curve cryptography - ECC}

Elliptic curve cryptography is a branch of public key cryptography, a.k.a asymmetric-key cryptography. An elliptic curve is defined by a curve equation built on a finite field F for coordinates. In order to understand elliptic curves, knowledge about group theory and number theory are required. In the next section, we will introduce the finite field and an interesting characteristic of the group that the elliptic curve used. Then, we explain the elliptic curve cryptography, the ECDLP and some case studies that are necessary for our thesis.

\subsubsection{Finite field \& cyclic group}

A finite field, sometimes called Galois field, is a set with a finite number of elements. Roughly speaking, a Galois field is a finite set of elements in which we can add, substract, multiply and invert. Before we introduce the definition of a field, we first need the concept of a simpler algebraic structure, a group

\textbf{Definition:} A group is a set of elements $G$ together with an operation $\circ$ which combines two elements of $G$. A group has the following properties
\begin{itemize}
  \item The group operation $\circ$ is closed. That is, for all $a,b \in G$, it holds that $a \circ b = c \in G$

\end{itemize}

So, a group is set with one operation and the corresponding inverse operation. If the operation is addition, the inverse operation is subtraction; and if the operation is multiplication, the inverse operation is division or multiplication with the inverse element. The number of elementsWhat we call a field is a group with all four basic arithmetic operations. A field is defined as


In the elliptic curve, we are interested in fields with a finite number of elements, the finite field. The number of elements in the field Fp is called the order (cardinality) of the field, denoted by |Fp |. A field with order m only exists if m is a prime power, i.e., m = pn, for some positive integer n and a prime integer p. p is called the characteristic of the finite field, denoted as char(p). A finite field is denoted as GF(pn) or Fpn. The most intuitive examples of finite fields are fields of prime order with n = 1. Elements of the prime field Fp can be represented by integers 0, 1, …, p-1. The two operations of the field are modular integer addition and integer multiplication modulo p. With n > 1, the field is called extension field, the elements are polynomials an-1Xn-1 + an-2Xn-2 + … + a1X + a0, where the coefficient a is in Fp. The arithmetic operations in an extension field are polynomial addition modulo p with the coefficients and polynomial multiplication modulo P(x). P(x) is called the irreducible polynomials, which acts like ‘a prime’ of the polynomials.


\subsubsection{Weierstrass curve - the general form of an elliptic curve}


\subsubsection{Elliptic curve discrete logarithm problem - ECDLP}

Given an elliptic curve E, we consider a primitive element P and other element T, the discrete log problem is finding an integer d, where

% d ∈[1, #E] such that P + P + … + P (“+” d times) = dP = T.
The ECDLP is considered to be infeasible if the elliptic curve parameters are carefully chosen [Selecting elliptic curve for cryptography: an efficiency and security analysis]. In ECC, we interpret the integer d as the private key, T as the public key, and the exact number of point points E of a curve. The number of points on the curve can be counted by trying all the possible values for x in Fp, but this is not efficient if the p is a large prime. Haskel’s theorem only gives an upper and lower bound for number of point on an elliptic curve

\subsubsection{Koblitz curve, case study: Secp256k1}

\subsubsection{Montgomery curve, case study: Curve25519}

\subsubsection{Edwards curve and the twisted Edwards curve, case study: Edwards25519}


\subsection{Digital signature with ECC}

\subsubsection{The standard ECDSA}

\subsubsection{Schnorr’s signature scheme and EdDSA}

\subsection{Comparison between ECDSA with Secp256k1 and Ed25519}

\subsubsection{Secp256k1 vs. Ed25519}

\subsubsection{ECDSA vs. Ed25519}


\subsection{Pseudo Random Number Generator}


\subsection{Cryptographic hash function}

\subsubsection{Secure Hash Algorithm with digest 512 bits (SHA-512)}

\subsubsection{Hash-based message authentication codes (HMACs)}
