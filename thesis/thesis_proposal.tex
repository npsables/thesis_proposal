\documentclass[12pt,a4paper,oneside]{book} % twoside for draf

%\usepackage{times}
\usepackage{graphicx}
\usepackage{subcaption}
\usepackage[utf8x]{vietnam}
\usepackage[english]{babel}
\selectlanguage{english}
\usepackage{mathptmx}	% same Time New Roma
%\renewcommand{\rmdefault}{phv} % Arial
%\renewcommand{\sfdefault}{phv} % Arial
\usepackage{url}

%link to part by click on contents region
\usepackage{hyperref}
\hypersetup{
    colorlinks,
    citecolor=black,
    filecolor=black,
    linkcolor=black,
    urlcolor=black
}

\usepackage{fancyhdr}
\usepackage{algorithm2e}
\usepackage{amsmath}
\usepackage{array,tabularx}
\usepackage{amsfonts}
\usepackage{amssymb}
\usepackage{cases}
\usepackage{tabularx}
\usepackage{adjustbox}
\usepackage{multirow}

\usepackage{bkthesis}

\usepackage{titlesec}

%\titleformat{\chapter}[display]
%  {\normalfont\bfseries}{}{0pt}{\Huge}

\usepackage{tikz}
\usetikzlibrary{shapes,arrows}

\newenvironment{conditions*}
  {\par\vspace{\abovedisplayskip}\noindent
   \tabularx{\columnwidth}{>{$}l<{$} @{${}:{}$} >{\raggedright\arraybackslash}X}}
  {\endtabularx\par\vspace{\belowdisplayskip}}

\tikzstyle{blank} = [text badly centered, node distance = 2cm, inner sep=0pt]
\tikzstyle{block} = [rectangle, draw, 
    text width=2em, text centered, minimum height=2em]
\tikzstyle{line} = [draw, -latex']

\newcolumntype{b}{>{\hsize=1.0\hsize}X}

\crname{THESIS PROPOSAL}
\ctname{Name of THESIS}
\cstuname{
	Students: \\
	\begin{tabular}{ l c }
	    Nguyễn Nguyên Phương & 1712726\\
		Nguyễn Đình Thắng & 1752048
	\end{tabular}
}


\csCouncil{Computer Science}
\csSupervise{Nguyễn An Khương}
\csReviewer{No name}
\cttime{12/2020}

\thesislayout

\begin{document}

\coverpage

\frontmatter

\begin{declaration}
We hereby undertake that this is our own research project under the guidance of Dr. Nguyen. Research content and results are truthful and have never been published before. The data used for the analysis and comments are collected by us from many different sources and will be clearly stated in the references.

In addition, we also use a number of reviews and figures of other authors and organizations. All have citations and origins.

If we detect any fraud, we take full responsibility for the content of our graduate internship. Ho Chi Minh City University of Technology is not related to the copyright and copyright infringement caused by us in the implementation process.

\end{declaration}
~

\begin{acknowledgments}

First and foremost, we would like to express our sincere gratitude to our advisor Dr. Nguyen for the support of our thesis proposal for his patience, enthusiasm, experience and knowledge. He shared his experience and knowledge which helps us in our research and how to provide a good thesis proposal.
\end{acknowledgments}
~

%
\begin{abstract}
Charts have been and always will be one of the most effective tools for demonstrating and sharing ideas among others. Besides text and images, drawing flow charts is the best way to give others a clearer path of the plan with the least amount of work. Nowadays, many meetings require a blackboard so everyone can express their thoughts on. This raised a problem with saving these drawings as a reference for future use since taking a picture of them will not solve the problem of re-editing these ideas and they need to be re-drawn to be suitable in professional documents. On the other hand, in order to digitalize the chart required to re-draw it using a computer or a special device like drawing boards and digital pens, which cost a lot and is not the most convenient tools to use.

Therefore, it is necessary to find a new way to convert the current hand-drawing charts into digital ones effortless, simplify the sharing process between users and be able to export them into another form like picture files (png, jpg), document files (pdf) or common diagram editing files (drawio). The application must be able to run on popular platforms and accessible by everyone.
\end{abstract}
 
\tableofcontents
%\listofsymbols
\listoftables
\listoffigures
%\listofalgorithms


\mainmatter

\fancypagestyle{plain}{%
\fancyhf{}
\fancyfoot[LE, RO]{\thepage}
\renewcommand{\headrulewidth}{0.0pt}
}

\pagestyle{fancy}
\renewcommand{\chaptermark}[1]{\markboth{\MakeUppercase{#1}}{}}
\fancyhf{}
\fancyhead[LE,RO]{\leftmark}
\fancyhead[RE,LO]{}
\fancyfoot[LE,RO]{\thepage}
\renewcommand{\footrulewidth}{0.4pt}

%\fancyhead{}  % Clears all page headers and footers
%\rhead{\thepage}  % Sets the right side header to show the page number
%\lhead{}  % Clears the left side page header
%\fancyfoot[positions]{footer}
%\renewcommand{\footrulewidth}{0.4pt}

%\pagestyle{fancy}  % Finally, use the "fancy" page style to implement the FancyHdr headers


\chapter{Introduction} 
\label{Introduction}

\textit{In this chapter, we are going to discuss the overview of the wallet we are creating. Thus, we are going to present the objectives, scope, and structure of this thesis.}
\adjustmtc
\minitoc
\section{Overview}
\label{overview}
\bigskip

At the moment, blockchain is a robust trend in financial technology. Blockchain is atechnology that enables truthless digital currencies transaction without intermediaries. It helps to prevent the double-spending and solve the censorship problem in traditional finance. Similar to a transaction in a bank, you need to sign it to provide the proof of identity. In the blockchain system, you use a digital signatures schema. The private key is kept private and used to sign the transaction, while the public key is accessible by everyone, used to verify the signature. The sender can also use your public key as a public address to send cryptocurrency.

The pair of keys is preserved and managed by using a cryptocurrency wallet, or crypto wallet for short. Some people may call it a digital wallet, but that is not entirely correct. A digital wallet (e-wallet) such as Momo and Paypal can hold users’s digital assets (government-issued currency) and link to their credit/debit cards. The banks apply the online banking service and allow the digital wallet to connect to users’s accounts, which means users are required to sign up the service first to use a digital wallet and they have to transfer indirectly through the banking system. On the other hand, crypto wallet has access to users’s crypto assets, and can help them to transfer cryptocurrencies directly because the blockchain system does not required a trusted third party nor any intermediary. All the transfer process takes in terms of minutes, no matter how far it is from you to the receiver. With some lightning-fast consensus like the Proof of History blockchain, it took less than 10 seconds. In the case of digital wallets or bank accounts, you have to wait for days before your transaction is confirmed, and it even worse when you decide to transfer to another country/state or cross-bank. The bank does have a solution for this problem (NAPAS, Visa, MasterCard,...) but you have to pay an extra fee for the service no matter how much you spend. That is why rather than the traditional stock markets, people are more likely to invest in the decentralized system since they have to stay home due to the COVID pandemic. The cryptocurrency still has some more advantages to traditional finance, but for the scope of our thesis, we won’t include them here. Another difference is that, crypto wallet does not hold any of users’s private information such as phone number, name, ID card, etc (also know as KYC). Therefore; users’s information is protected and will not be leaked even if the crypto wallet got hacked.

In practical, the investors do not invest in a single market. They love to split their funds and put their money into multiple projects because one of the famous investment rule is ``Don’t put all your eggs in one basket". However; a crypto wallet creates a different key pair to sign for each transaction and each crypto wallet is connect to only one platform. The users have to create multiple wallet for each market, and they have to keep track of the used key pairs for both backing up and avoiding key leakage. This is inconvenience for both advanced users and investors since hundreds of transactions are made and hundreds of markets launch per day.

Also, crypto wallet or digital wallet, they are still online applications that rely on the Internet. An advanced user acknowledges that the blockchain system and the internet is not a safe place. Both of the wallets are vulnerable to common attacks such as man-in-the-middle attack, phishing, social engineering, and more. These attacks are hard to prevent because they focus on the human factor: the users’s awareness. In addition, blockchain system is a collection of digital signature, which reckons the digital signature algorithm (DSA). If users lost their private key, or the cryptography behind the DSA is broken, all of their assets are gone and there are no way to retrieve those assets because blockchain system holds no personal information about the users.

\section{Objectives}
\label{objectives}

To solve the first problem, we choose to work on an hierarchical deterministic wallet (HD wallet). The HD secret key derivation and transfer protocol allow creating a huge amount of child keys from parent keys in a hierarchy. Crypto wallets using the HD protocol are called HD wallets. HD wallet helps the users to manage multiple child wallets and backup easier. If a child wallet is lost, the other child wallets remain safety when the parent keys are not leaked.

For the second problem, the main reason why we brought it up is that we found the standard digital signature algorithm (ECDSA) used in the blockchain system has a flaws since 2009 \cite{Schmidt2009} so the need for a higher-security signature is vital. What catches our eyes is the Ed25519 digital signature schema. The release of curve Curve25519 \cite{Bernstein2006} from the SafeCurves project \cite{safecurves}, which has an incredible speed and is applicable in the digital signature Ed25519 \cite{Bernstein2011}, marks a new trend in elliptic curve cryptography (ECC). Also, the expiration of Schnorr’s signature patent in 2008 allows Ed25519, a digital signature based on Schnorr’s signature and the twisted Edwards curve, to be publicly used. The advantages of Ed25519 are comprised in the RFC 8032 document \cite{Josefsson2017}:

\begin{enumerate}
    \item High performance in various platforms.
    \item Generating a random number for each signature is not needed.
    \item Resilience against side-channel attacks.
    \item Complete formulas (e.g. addition law): the formula works for all points in the curve.
    \item Hash collision resistance.
    \item 128-bit security level.
    \item Small 32 byte keys and 64 byte signatures
\end{enumerate}
This is the motivation for our thesis, to develop an HD wallet for Ed25519. We will explain the benefits of the HD wallet, along with implementing and analyzing the Ed25519 of HD wallet’s usage. 

\section{Scope of the study}

In this thesis, our main focus is to study elliptic curve cryptography and its digital signature algorithm to point out the difference between the standard elliptic curve digital signature algorithm (ECDSA) and the Ed25519 signature schema in a cryptanalytical way. For blockchain, we take a look at how to make a transaction using crypto wallet and how a crypto wallet is connected to a blockchain system. Finally, we try to build a prototype for an HD wallet using Ed25519 signature schema, limited to a web application, with an open-source library and the application will support at most 3 blockchains. We look for open-source protocols and choose Solana blockchain as our testing environment because Solana uses Ed25519 for their signatures.

\section{Thesis structure}

The contents of our thesis are presented in seven chapters, namely:

\textbf{Chapter 2} introduces and discusses the background knowledge of this thesis, including the necessary knowledge of the HD wallet and blockchain network, group and field for elliptic curve cryptography, schemas of Ed25519 and its performance as well as the important cryptography functions.

\textbf{Chapter 3} discusses the protocols of the HD wallet, the related works on implemented library and some open source wallet that used in the industry. 

\textbf{Chapter 4} shows our system design for the HD Wallet for Ed25519, how did we solve the challenge on key derivation schema. We also improve the wallet by letting it hold tokens of different blockchains (expanding with Secp256k1 curve).

\textbf{Chapter 5} describes how we actually implement the HD wallet, including the discussion on library that we used.

\textbf{Chapter 6} summarizes the testing results of our application.

\textbf{Chapter 7} concludes our work. Also, we analyze the drawbacks and how to prevent it with the spaces for improvements to the thesis.

\chapter{Background}
\label{chap:background}
	\textit{In this chapter, we introduce the foundation knowledge of the thesis, including the history and definition of Blockchain Technology, Cryptocurrency, 
	Hierarchical Deterministic Wallet (HD Wallet) and Cryptography}
\minitoc

\section{Blockchain Technology}

\subsection{History and Definition}

Blockchains are immutable digital ledger systems implemented in a distributed fashion (i.e., without a central repository) and usually without a central authority.
The definition of blockchain was introduced to the world by a person (or a group of people) under the name Satoshi Nakamoto on October 31, 2008. 
It was applied to enable the emergence of a "purely peer-to-peer (no financial institution or third party) electronic cash" named Bitcoin where transactions take place in a distributed system.
In fact, Satoshi did not invent blockchain, and Bitcoin blockchain is not the first chain that ever created. 
Back in 1991, cryptographers Stuart Haber and Scott Stornetta published a whitepaper "How to Time-Stamp a Digital Document" in the Journal of Cryptography. 
Their goal is to digital time-stamping of documents so that it is infeasible for a user either to back-date or to a forward-date digital document, even with the collusion of a time-stamping service. 
The technology is called a blockchain because the distributed electronic ledger stores items of data in time-stamped digital groups called blocks. Each block includes an alphanumeric code called a "hash" summing up its data. The hash of each completed block also appears in the next one in the chain, which means that to alter one block you would have to alter all the ones connected to it. These cryptographic dominos function together to protect against tampering or fraud.
Base on this theory, the longest-running blockchain, started in 1995, also by Haber and Stornetta, publishes the weekly summary hash value every week in the New York Times (\autoref{fig:first_blockchain}) and still running strong today. 

\begin{figure}[h!]
	\centering
	\includegraphics[width=.35\textwidth]{images/Widely_Witnessed_Values.jpg}
	\caption[Widely-Witnessed Values of Surety, a weekly summary (hash) of documents]{Weekly summary hash value in The New York Times}
	\label{fig:first_blockchain}
\end{figure}

% The math behind blockchain and its complex system architecture make it challenging to understand. 
But the word "blockchain" or "block" and "chain" wasn't use back then. 
Only it become known in Satoshi Nakamoto's Bitcoin paper in the term of "chain" of "blocks".
Later people combined the one-word "blockchain" in mainstream media publications such as Fortune, Forbes, and the Huffington Post as the technology gained greater interest and use. 
The community use that word for Nakamoto's invention.
Bound to the emergence of Bitcoin and cryptocurrency, a concise description of blockchain technology is provided by NIST:

\begin{quote} 
	Blockchains are distributed digital ledgers of cryptographically signed transactions that are grouped into blocks. Each block is cryptographically linked to the previous one (making it tamper evident) after validation and undergoing a consensus decision. As new blocks are added, older blocks become more difficult to modify (creating tamper resistance). New blocks are replicated across copies of the ledger within the network, and any conflicts are resolved automatically using established rules.
\end{quote}

Blockchain technology comes handy in a wide range of areas - both ​financial and non-financial​. 
Non-Financial application opportunities are endless. 
We can envision putting proof of the existence of all legal documents, health records, and loyalty payments in the music industry, notary, private securities and marriage licenses in the blockchain. 
By storing the fingerprint of the digital asset instead of storing the digital asset itself, the anonymity or privacy objective can be achieved.
For the sake of our thesis, we will mainly focus on the original and surely the most popular application of blockchains - Cryptocurrency.

Cryptocurrencies are digital currencies that use blockchain technology to\autoref{fig:first_blockchain}) and still running strong today. 
record and secure every transaction. 
A cryptocurrency can be used as a digital form of cash that can be used to buy goods and services. 
It can be bought using one of several digital wallets or trading platforms, then digitally transferred upon purchase of an item, with the blockchain recording the transaction and the new owner. 
The appeal of cryptocurrencies is that everything is recorded in a public ledger and secured using cryptography, making an irrefutable, timestamped, and secure record of every payment.
The ledger displays user account balances and inter-user payments in a “currency” defined by the ledger itself and not necessarily in one of the traditional currencies. 
Nevertheless, cryptocurrency may be traded on the stock exchange and exchanged for traditional money, which makes it hard to distinguish between traditional currency and cryptocurrency and as official vs. non-official currency. 
The most widely recognized cryptocurrency system is Bitcoin.

We believe the "magic" that brings the above concept of digital currencies to reality, besides blockchain technology, is Nakamoto's proof-of-work consensus model.

\subsection{Blockchain Characteristics and Generations}

Blockchain network can be:
\begin{itemize}
	\item \emph{Permissioned network}, where users publishing blocks must be authorized by some authority (be it centralized or decentralized). 
	Users of blockchain have to trust that entity or user who published blocks. 
	Permissioned blockchain networks may thus allow anyone to read the blockchain or they may restrict read access to authorized individuals. This maybe used by organizations that need more control over their blockchain.
	Some permissioned blockchain networks support the ability to selectively reveal transaction information based on a blockchain network users identity or credentials. 
	Some of famous permissioned blockchain applications are Ripple, which enables interbank transactions, or Sovrin, which is managed by financial institutions and is seeking to build a global decentralized identity system.
	
	\item \emph{Permissionless network}, where service providers are not fixed and, in principle, anyone can start operating the service. 
	For example, Bitcoin and the early versions of Ethereum.
	
	\item \emph{Centralized network}, where the ledger is managed as a centralised service by one legal	entity. For example, the Guardtime system.
\end{itemize}

% The blockchain is usually stored and managed in the form of a distributed ledger,
% with multiple parties keeping a copy of the ledger, which then implies the use of a
% handshake protocol between the components. However, also centralized blockchain
% system exist. The original and surely the most popular application of blockchains is
% cryptocurrency, where the ledger displays user account balances and inter-user payments
% in a “currency” defined by the ledger itself and not necessarily in one of the traditional
% currencies. Nevertheless, cryptocurrency may be traded on the stock exchange and
% exchanged for traditional money, which makes it hard to distinguish between traditional
% currency and cryptocurrency and as official vs. non-official currency. The most widely
% recognised cryptocurrency system is Bitcoin, which establishes and uses Bitcoins and
% Satoshis as currency.

% Most permissionless blockchain systems include an independent cryptocurrency. 
% The reason for that is that in the absence of an inter-operator contract, there are usually no other incentives to guarantee voluntary management of the Blockchain

Based on the intended audience, three generations of blockchains can be distinguished (Zhao et al., 2016):
\begin{itemize}

\item Blockchain 1.0 which includes applications enabling digital cryptocurrency transactions
\item Blockchain 2.0 which includes smart contracts and a set of applications extending beyond cryptocurrency transactions
\item Blockchain 3.0 which includes applications in areas beyond the previous two versions, such as government, health, science and IoT.

\end{itemize}

We are now developing blockchain 2.0 but our thesis just focuses on the cryptocurrency aspect.


\subsection{Bitcoin blockchain}
Bitcoin is the first application of blockchain and the most famous digital currency ever.
As mentioned above, Bitcoin was invented with the publication of a document entitled "Bitcoin: A peer-to-peer electronic cash system" in 2008 by Satoshi Nakamoto, mentioned as a purely P2P version of electronic cash would allow online payments to be sent directly from one party to another without going through a financial institution. 
The currency began to use in 2009 when its implementation was released as open-source software. 
The Bitcoin blockchain is considered to be a world-changing technology because in the first time in human history its solved the biggest problem of distributed system: The Byzantine General's Problem. 
We will talk about this in the Bitcoin game of theory and incentives section.

Bitcoin application is one of the permissionless blockchain.
It utilize well-known computer science mechanisms (linked lists, distributed networking) as well as cryptographic primitives (hashing, digital signatures, public/private keys) mixed with financial concepts (ledgers, games of theory) in high level. 
Base on the problems Bitcoin has solved, we examine by dividing it into 3 components:

\begin{itemize}
\item Secure and Prevent tempering the data

\begin{quote} 
	\emph{Hashes} - 
	Cryptographic hash functions (CHF) are used for hashing the content of a block, validating the integrity of data, reduce the size of the message or keys, generating a Bitcoin address. We will show detail at Section~\ref{sec:crypto_hash}.
	Hashing is a method of calculating a relatively unique fixed-size output (called a message digest, or just digest) for an input of nearly any size (e.g., a file, some text, or an image).
	Even one single bit change of input will result in a completely different output digest. 
	In Bitcoin and most blockchain technologies, SHA-256 (Secure Hash Algorithm with output size of 256 bits) appear the most. Many computer support hardware level for this algorithm.
	NIST specified this algorithm for SHA-256 in Federal Information Processing Standard (FIPS) 180-4 as it passed every properties of a cryptographic hashing.
	\autoref{fig:example_of_sha-256} is an example of SHA-256.

		\begin{figure}[h!]
			\centering
			\includegraphics[width=1\textwidth]{images/example_of_sha-256.png}
			\caption[Example input and output of SHA-256 Digest Value]{Example I/O of SHA-256 Digest Value}
			\label{fig:example_of_sha-256}
		\end{figure}

	\bigbreak
	
	\emph{Public/Private Key} -	
	Asymmetric-key cryptography (or public-key cryptography) uses a pair of keys: a public key and a private key that are mathematically related. It could be infeasible to generate one key from the other.
	The private key is kept secret while the public key can be to everyone, both keys are hold inside user's Wallet, which we present in Section~\ref{sec:hd_wallet}.
	One can encrypt with a private key and then decrypt with the public key. 
	Alternately, one can encrypt with a public key and then decrypt with a private key.
	Bitcoin uses asymmetric-key cryptography to digitally sign transactions, verify signatures or in some cases, exchange the key.
	Asymmetric-key cryptography is discussed in Section~\ref{sec:asymmetric_cryptography}.
	Figure~\ref{fig:asymmetric_cryptography} briefly show message exchange usage of the asymmetric protocol.
	
	\begin{figure}[h!]
		\centering
		\includegraphics[width=0.4\textwidth]{images/asymmetric_cryptography.png}
		\caption[An example of concept of Asymmetric-key cryptography]{Sending private message using Asymmetric-key cryptography}
		\label{fig:asymmetric_cryptography}
	\end{figure}
	\bigbreak

	\emph{Transactions} - Transactions represent transfers of the cryptocurrencies between wallets in the system. 
	A transaction contains input and output. The inputs are usually a list of the digital assets to be transferred.
	Outputs are	the accounts that will be the recipients of the digital assets along with how much digital asset they will receive.   
	The  output  of  a  transaction  is categorized  by  either  unspent  transaction  output  (UTXO)  or spent  transaction  output  (STXO)
	All values of in and out cannot be tampered with.

	\begin{figure}[h!]
		\centering
		\includegraphics[width=0.7\textwidth]{images/transaction.png}
		\caption[An example of bitcoin transaction]{An example of bitcoin transaction}
		\label{fig:transaction}
	\end{figure}
	\bigbreak

	All transactions are broadcast to the network and usually begin to be confirmed within 10-20 minutes, through a process called \emph{mining}.
	Transactions are typically digitally signed by the sender’s associated private key and can be verified using the associated public key.

	\bigbreak

	\emph{Ledgers} - 
	A ledger is a collection of cryptographic transactions. 
	Bitcoin ledgers are distributed, the blockchain holds all accepted transactions within its ledgers. Every user can maintain their own copy of the ledger.
	Whenever new full nodes join the blockchain network, they reach out to discover other full nodes and request a full copy of the blockchain network’s ledger, making loss or destruction of the ledger difficult.
	\bigbreak
	
	The network utilizes cryptographic mechanisms such as digital signatures and cryptographic hash functions to provide tamper-evident and tamper-resistant ledgers.
	Due to the public distributed network, the Bitcoin blockchain is harder to attack. There is nothing to steal because everything is distributed. If one individual node got taken down, the network will still be running. 
	If targeting the blockchain itself, the attackers will face resistance from the honest nodes present in the system. 
	\bigbreak

	\emph{Blocks} -	
	Transactions, after sent to the network (by wallets, web applications, etc.), will be, if accepted, added to a block that is published by a chosen node. 
	Bitcoin blocks include block header and block data.
	Figure~\ref{fig:block_component} show basic component of a block.
	Block header contains Height, previous block header’s hash value (prevBlockHash), a hash representation of the block data (usual Merkle tree* hash), a timestamp, size of the block (bits), a \emph{nonce}, etc.
	The \emph{nonce} value is manipulated by the publishing node to solve the hash puzzle (see Game of theory \ref{item:game_theory})
	Block data contains a list of transactions and ledger events. Some include other data.
	\pagebreak

	\begin{figure}[h!]
		\centering
		\includegraphics[width=0.6\textwidth]{images/block_component.jpg}
		\caption[Components of Bitcoin block]{Components of Bitcoin block}
		\label{fig:block_component}
	\end{figure}

	\emph{Chain of Blocks} - 
	Blocks are chained together through each block containing the hash digest of the previous block’s header, thus forming the blockchain.
	If one of the previous blocks were changed, it would result in a different hash.
	This makes it possible to easily detect and reject altered blocks

	\begin{figure}[h!]
		\centering
		\includegraphics[width=1\textwidth]{images/chain_of_block.png}
		\caption[Components of Bitcoin block]{Components of Bitcoin block}
		\label{fig:chain_of_block}
	\end{figure}

\end{quote}

\item Game of theory and Incentives
\label{item:game_theory}

	Arthur C. Clarke once wrote, “Any sufficiently advanced technology is indistinguishable from magic”.
	Clarke’s statement is a perfect representation for the emerging of Bitcoin. 
	Bitcoin blockchain is considered to be a masterpiece of combination of mordern technologies, economics, mathematics, philosophy, and game theory.
	What make it so special is applying game of theory to solve the The Byzantine General’s Problem.
	
	Game theory is based on the assumption that all participants are rational actors and are trying to maximize their gains from the game.
	In the situation of Bitcoin, the problem is how to ensure that all the actors of a decentralized network behave correctly. 
	This is known as The Byzantine General’s Problem.
	How to ensure that all generals will follow the plan? Even if they are located in different places and do not trust each other?
	This was believed to be the biggest problem of P2P network, impossible to achieve before Satoshi Nakamoto.

	To make Bitcoin work, Nakamoto invented a model called Proof of Work Consensus Model. 
	In simple words, the proof-of-work involves incrementing the \emph{nonce} in the block that when hashed with SHA-256, the hash begins with a number of zero bits (number of zero bits called the difficulty). 
	User publishes the next block by being the first to solve this puzzle, such users are miners in Bitcoin.
	Finding a block with a specified difficulty is called mining. When successfully mined a new block, the miner gets some Bitcoin as a reward, this is called \emph{Incentive}. 
	The \emph{Incentive} help encourage nodes to stay honest.
	The accepted chain among the system is the longest chain, which has the greatest proof-of-work effort invested in it.
	If a majority of honest nodes control the power, the honest chain will grow the fastest and outpace any competing chains.
	Proof-of-work aims to protect the integrity of the ledger by making it hard to modify transactions in "old" entries, as changing a block would also require changing all subsequent (later) blocks.

\item Communication Network	
	\begin{quote} 
		Bitcoin blockchain network is peer-to-peer relying on consensus PoW model, where peers (or nodes) are equal right in terms of authority. 
		New nodes can join at any time. The connection establishes over TCP, generally on port 8333, an ad-hoc network with random topology. 
		The system is designed to eliminate third party in certifying transactions in term of trust and reliability.
		This drives step from a centralized to a decentralized solution.
		Also, as the nature of decentralized network, it will work even if a single point of network failure.
		A problem of Cryptocurrency is double spending attacks. A malicous user can spend his asset on multiple transactions.
		For distributed network to accomplish this, all transactions have to be publicly broadcast through the network. 
		The payee needs proof that at the time of each transaction, the majority of nodes agreed it was the first received through consensus agreement.
		How ever if an attacker is able to control at least 51\% of the has power of the network, they can commit double transaction and verify it.
		This attack is feasible now since more than 60\% of mining power are located in China.
		
	\end{quote}
\end{itemize}



\section{HD Wallet}
\label{sec:hd_wallet}

\subsection{Category}

\subsection{Coins}

\subsection{Wallet structure}

\section{Cryptography}

\subsection{Cryptographic hash}
\label{sec:crypto_hash}

\subsection{Digital Signatures}
\label{sec:digital_signatures}

\subsection{Asymmetric-key cryptography}
\label{sec:asymmetric_cryptography}

\subsubsection{Diffie-Hellman algorithm}

\subsubsection{RSA Cryptography}

\subsubsection{EC Cryptography}

\subsection{Twisted-Edward curve and Ed25519}

\subsection{Key derivation function}

\chapter{Methodologies and Approaches} \label{chap:Methodologies_and_Approaches}

\section{Challenges}

\subsection{HD wallet architecture for Ed25519}

\subsection{Key managements}

\section{Approaches}


% \begin{itemize}
%     \item The topic focuses on building mobile apps on Android platform using Flutter and the main programming language is Dart and doesn't support other mobile platforms like iOS, Windows Phone.
%     \item User can only convert hand drawn diagram to digital version if internet connection is available.
%     \item The web server can serve 100 clients at the same time.
%     \item The system only supports converting and editing of small charts (in an A4 page, medium size text) and does not support importing files from other platforms.
%     \item The system only create and editing flowchart.
%     \item The system only supports storing chart files in .json format and exporting to images (png, jpg), text (pdf) and .drawio files.
%     \item The flowchart contains maximum 20 symbols not including arrows
%     \item The system do not allow to take picture with the wrong direction.
%     \item The flowchart must not have three or more arrows intersect at one point. It also must not have two arrows having the same edge.
%     \item The flowchart must be drawn with ball pen on a fresh white paper.
%     \item Arrow in flowchart must have an arrowhead. 
%     \item There must not be isolated text/symbol. Furthermore, strikethrough text is restricted.
% \end{itemize}

% \section{Application features}

% The application runs on mobile devices so that users can easily convert a graph on the fly as long as they are connected to the internet. It also requires storage access and camera access permission in order to use the scanning function and temporarily saving it in the phone.

% \subsection{Sign up and Login}
% When users open the app for the first time, they will be offered to create an account to continue. To simplify the sign-up process, it only requires the user's email and password. Other information can be updated later on. In addition, users can also use their Google account to sign in without creating their own account for this app. In case users had already forgotten the password, they can retrieve it by providing their account email and the system will send an email to verify and include further information for changing the password.

% \subsection{Create diagram}
% After successfully login in, users can create a new diagram using one of these methods:
% \begin{itemize}
%     \item Directly Scanning: The application activates the device's camera so user can point at the drawing and take a picture of it.
%     \item Import from image: User selects an image from device's gallery.
%     \item Create blank: User can create a blank diagram and add elements then.
% \end{itemize}

% \subsubsection{Convert to digital}
% If user chooses the first two options then after taking/choosing the picture he/she can re-take or re-adjust the picture (rotate, flip, or crop) before finishing this step. After that, the result will be displayed and provide some extra options: save, modify, and delete. Save option will require inputting the diagram name and choose the saving location. The diagram can be saved online on the web server or offline on the phone. Each user can only save a limited number of diagrams online, if a user is running out of saving slot then the only option is to store them locally. The main difference between saving online and offline is only the diagrams on the web server can be shared and have a detailed change history. Modify option is used if user is not satisfied with the result of the auto-detection algorithm and want to change it. This option will send them to the modify page where they can edit it before saving. Note that if user leaves this page without saving it, the diagram will be lost. Delete option shows a confirmation message and offers a retry.

% \subsubsection{Create from blank}
% If ``Create blank'' is chosen then a name for new diagram is required, then user will be sent to modify page where already has a default diagram created.

% \subsection{Adjust diagram}
% All the created diagrams will be displayed on the main screen of the app, user can select one of them to modify, delete, rename, view history change, export, and share. Only diagrams that are stored online have the share option and view history change option.

% On the modify page, user can change elements' position, create new ones, or delete them. All elements supported will be listed later. All the attribute that user can change within an element is size, title, color, and type of element. All of the changes will be stored so that the user is able to undo or redo the action. Auto-saving is not supported, user needs to save manually. If user exits the app without saving, all of the progress will be lost.
% "Change history" function requires to store a version of the old diagram every time a new one is saved into the web server. If the diagram is shared within a group, all of the information about the user who makes changes is also stored.

% \subsection{Share and set permission}
% Users can also share the diagram with others. However, only the owner can share their product. All of the shared diagrams will be in another tab on the home screen. To share a diagram, user needs to input the receiver's email and set their permission to "Read-only" or "Can edit". The Owner can also revoke the share permission by deleting the other user from the share list.

% \section{Requirements}
% \subsection{Functional requirement}
% \begin{itemize}
%     \item {System is able to detect the flow-chart diagram from a hand-drawing draft.}
%     \begin{itemize}
%         \item {System is able to detect the node type and its title.}
%         \item {System is able to detect the relationship between the diagram’s symbols and check for incorrect relationship.}
%         \item {System is able to detect the relationship between the diagram’s symbols and check for incorrect relationship.}
%         \item {English is the main supported language.}
%     \end{itemize}
%     \item {System is able to re-create the diagram into digital version to display in the mobile application, and allow user to modify.}
%     \item {The input can be directly taken from the camera of the device or uploaded from the storage.}
%     \item {(Optional) System has the ability to convert the diagram to some popular diagram file format so that user can import it from tools like drawio, lucidchart, …}
% \end{itemize}

% \subsection{Nonfunctional requirement}
% \begin{itemize}
%     \item {System is written with Dart (with framework Flutter), C++.}
%     \item {System is able to detect diagram within 8 seconds.}
% \end{itemize}

% \subsection{Hardware requirement}
% \begin{itemize}
%     \item {Mobile application require device must be installed with android OS version 7.0 or above }
%     \item {Recommend requirement for web server:}
% \end{itemize}
\chapter{Goals} \label{chap:Goals}




% \section{System Architecture}
% \begin{figure}[!b]
% \includegraphics[width=14cm]{Images/App/System_Design.png}
% \caption{System Architecture Design}
% \end{figure}

% At first, this application has two system options: One that requires no additional server to run, all of the service is installed locally on the user's devices. Since our core service, the diagram detection algorithm is written with C++ then it can run on any system. Then we intended to build an application system that can handle the whole process of pre-processing image and create the result without the internet connection. However, as our target is to support all the device that runs android 7.0 and above (that means all smartphones manufactured from 2016 to present and some of them can be older), it could lead to a performance problem because many of them were outdated 2 or 3 years. Another issue is sharing between users will have to be dependent on a third-party service. Which is not convenient and may lower the user experience.

% The second one is to build a REpresentational State Transfer Application Programming Interface (REST API) service, run on a node.js server, which will handle all the requests sent by users and store all the data on a system database so that users can now store all of their data online without worry about their phone internal storage. the web server can be divided into three main components: Request handler, Service handler and Database handler. All of the detection and management will be mainly process in the second component, then all of the data will be transfer to a My SQL database, including user data and diagram file. However, all of the uploaded images that sent to the server to process will be then stored in the image storage separated from main database for additional data to improve detection algorithm in the future. The downside of this approach is now in order to use the app, the application has to be stably connected to the internet to make any changes to their data (create, edit, remove,...). Another challenge for us is to optimize the server so it can process a lot of access at the same time.



% \section{Framework}
% \subsection{Flutter}
% Flutter is a user interface toolkit developed by Google for building beautiful, natively compiled applications for mobile, web, and desktop from a single code base \cite{57}. Dart is the main programming language used in Flutter.

% One of the most beloved of Flutter that finally made it the framework we use in this thesis is it can create great and modernized user interfaces that really suit our design code for this project: simplistic but easy to use. Although the result can be used to build both android and iOS (potentially web) applications, the target of this project is to focus on one platform at first and continue to grow in the future. Despite being a recently released framework built for mobile development, Flutter provides a lot of features that speed up the development process as well as a growing community to share and learn.

% However, there is also a downside of it, since it was born to build an application that can run on many different platforms, the performance loss is inevitable compared to natively built applications. Even though Flutter has been optimized and guarantee 60 frames per second but it still heavily depends on developers to make good use of available performance resources. Another one is the application file built by Flutter tends to have a bigger file size and also consume a lot more storage to install. A normal solution for this is to reduce image resolution and use less graphic and animations in the app, but Flutter still shows a poorer result than the native app.

% \subsection{Nodejs}
% Nowadays, Node.js is one of the most popular JavaScript Run time Environment as it is an open-source, cross-platform, and powerful tool that is used for many server-side projects \cite{59}. One of the biggest advantages is using JavaScript, which supports asynchronous programming. In the Node.js server, all requests will be handle by a single process without creating any thread. When a process requires queries in the database, the system allows pausing it until the queries are finished, in the meantime, other jobs can be handled without wasting a whole thread. As the number of clients increases over time, this can be a great feature to scale up the system.

% \section{Database Design}

% For each user, besides user name, email and password are essential information, an account must also have an account type to indicate whether it is a system account or from another platform account (such as Google). If so, it required another field to store the account ID key. If the account is created in the application then the password needs to be hashed and salted, which required 2 fields for salt string and encoded password. In order to perform any action with the server from the client-side, it requires a key for authentication each time the device calls an API, which will be changed each time a user logins. This key will be expired after half an hour of inactivity. Each time an API is called, the expired time will be refreshed. The expired time is stored with the user account for comparison each time this key is used. With this design, the system cannot keep track of the number of times a user logged in or how many times the API was called with a specific user key. However, all of the activities on the server will be recorded in other tables (creating and uploading diagram, send review), it will be unnecessary to create a table for login records.

% Each diagram has to belong to a user, diagrams can be shared between users with different permission like ''Owner'', ''Edit-Only'', and ''Read-only''.

% There can be two types of diagrams,  the ''converted'' diagram and the ''created from blank'' one. For the first type, it requires a field to store the image path in the image storage.

% As each diagram may have many versions, each time the user uploads a new one, it will be stored with a new ID and also have an upload type to distinguish between new upload and revert (as the user is able to view the change history, they may also revert back to an older version), then it will need an additional field to store the reverted version in order not to store many same version of the same diagram file.

% \begin{figure}[!b]
% \includegraphics[width=16cm]{Images/App/DB.png}
% \caption{Database Design}
% \end{figure}

% \section{Diagram File Design}
% After having consulted some types of diagrams, we decided to go with .json file since it is widely used and supported by many platforms. It is also able to be stored directly in the My SQL database using JSON data type. Although the size still depends on the server configuration, this is one of the most convenient file types to transfer between the web server and the client devices. The file will have 4 main components: ID, name, an array of symbols, and an array of arrows.

% Figure \ref{fig:jsonfile} shows the json file design used for construct the diagram used in the application.

% \begin{figure}[!b]
% \centering
% \includegraphics[width=12cm]{Images/App/jsonFile.png}
% \caption{JSON file design}
% \label{fig:jsonfile}
% \end{figure}



% \section{Use case Design}
% \subsection{Use case diagram}
% Figure \ref{fig:usecase} shows the system use case design for the application.

% \begin{figure}[!t]
% \centering
% \includegraphics[width=14cm]{Images/App/Usecase.png}
% \caption{Use case Design}
% \label{fig:usecase}
% \end{figure}

% \subsection{Use case detail}
% \begin{table}[b]
% \begin{tabular}{| m{8cm} | m{6cm} |}
% \hline
% Use case name & Table reference\\ \hline
% Login. & Table \ref{Tab:UC-1}\\ \hline
% Sign up. & Table \ref{Tab:UC-2}\\ \hline
% Create new diagram. & Table \ref{Tab:UC-3}\\ \hline
% Scan diagram with camera. & Table \ref{Tab:UC-3.1}\\ \hline
% Scan from image. & Table \ref{Tab:UC-3.2}\\ \hline
% Import file. & Table \ref{Tab:UC-4}\\ \hline
% Export file. & Table \ref{Tab:UC-5}\\ \hline
% Modify diagram. & Table \ref{Tab:UC-6}\\ \hline
% Delete diagram. & Table \ref{Tab:UC-7}\\ \hline
% % Share. & Table \ref{Tab:UC-8}\\ \hline
% % Set permission. & Table \ref{Tab:UC-9}\\ \hline
% % View change history. & Table \ref{Tab:UC-10}\\ \hline
% % Revert. & Table \ref{Tab:UC-11}\\ \hline
% Logout. & Table \ref{Tab:UC-12}\\ \hline
% \end{tabular}
% \captionof{table}{\label{Tab:UCList}Use case List}
% \end{table}

% \begin{table}[]
% \begin{tabular}{| m{4cm} | m{11cm} |}
% \hline
% Use case ID:       & UC-1 \\ \hline
% Use Case Name:     & Login. \\ \hline
% Triggering Event:  & The user open the app and in login screen. \\ \hline
% Brief Description: & System verify user's inputted username and password and go to home screen. \\ \hline
% Actors:            & The user \\ \hline
% Preconditions:     & \begin{itemize}
%     \item Application has internet connection.
%     \item The user has a system account.
%     \item The user has not logged in with Google account.
% \end{itemize} \\ \hline
% Post-conditions:   & \begin{itemize}
%     \item The user is logged in to the system.
%     \item The user is able to perform action that requires the web server.
% \end{itemize} \\ \hline
% Normal flows:      & \begin{enumerate}
%     \item The user opens the app.
%     \item System show login form including username and password.
%     \item The user fill in the form and login.
%     \item System authenticates user.
%     \item The user is logged in and has access to the system.
% \end{enumerate} \\ \hline
% Alternative flows: & \begin{itemize}
%     \item {2a The user has already logged in with Google account}
%     \begin{itemize}
%         \item 2a1 System automatically authenticates user in with Google account.
%         \item 2a3 Use case continue with step 5.
%     \end{itemize}
%     \item {3a The user chooses to log in with Google account}
%     \begin{itemize}
%         \item 3a1 System shows Google logging page.
%         \item 3a2 The user enters essential information.
%         \item 3a3 System receive Google credential.
%         \item 3a4 Use case continue with step 4.
%     \end{itemize}
%     \item {4a The user input an invalid username or password.}
%     \begin{itemize}
%         \item 4a1 System shows an error message.
%         \item 4a2 The user re-input information.
%         \item 4a3 Use case continue with step 4.
%     \end{itemize}
% \end{itemize} \\ \hline
% Exception: & \begin{itemize}
%     \item {2a The user logged in with Google account but the credential is incorrect.}
%     \begin{itemize}
%         \item 2a1 System shows an error message.
%         \item 2a2 Use case continue with step 1.
%     \end{itemize}
% \end{itemize} \\ \hline
% \end{tabular}
% \captionof{table}{\label{Tab:UC-1} Use case: Login.}
% \end{table}

% \begin{table}[]
% \begin{tabular}{| m{4cm} | m{11cm} |}
% \hline
% Use case ID:       & UC-2 \\ \hline
% Use Case Name:     & Sign up. \\ \hline
% Triggering Event:  & The user choose ''Sign up'' in login screen. \\ \hline
% Brief Description: & The user want to create a new system account. \\ \hline
% Actors:            & The user \\ \hline
% Preconditions:     & \begin{itemize}
%     \item The user has an email account.
%     \item Application has internet connection.
%     \item The user has not logged in the system.
% \end{itemize} \\ \hline
% Post-conditions:   & \begin{itemize}
%     \item A new account is registered in the system with distinct user name.
%     \item The user is logged in and has access to system function.
% \end{itemize} \\ \hline
% Normal flows:      & \begin{enumerate}
%     \item The user open the app and choose ''Sign up'' function.
%     \item System show the register form including username, email, password.
%     \item The user enters the essential information.
%     \item System check if all of the information are valid and create a new account.
%     \item System authenticates user.
%     \item The user is logged in and has access to the system.
% \end{enumerate} \\ \hline
% Alternative flows: & \begin{itemize}
%     \item {4a The user input an invalid username or password.}
%     \begin{itemize}
%         \item 4a1 System shows an error message.
%         \item 4a2 The user re-input information.
%         \item 4a3 Use case continue with step 4.
%     \end{itemize}
%     \item {4a The user input an already existed username.}
%     \begin{itemize}
%         \item 4a1 System shows an error message and provide login option.
%         \item 4a2 The user choose login.
%         \item 4a3 Use case continue with table \ref{Tab:UC-1}.
%     \end{itemize}
% \end{itemize} \\ \hline
% Exception: & N/A\\ \hline
% \end{tabular}
% \captionof{table}{\label{Tab:UC-2} Use case: Sign up.}
% \end{table}

% \begin{table}[]
% \begin{tabular}{| m{4cm} | m{11cm} |}
% \hline
% Use case ID:       & UC-3 \\ \hline
% Use Case Name:     & Create new diagram. \\ \hline
% Triggering Event:  & The user wants to create a new diagram. \\ \hline
% Brief Description: & At home screen, user wants to create a new diagram. \\ \hline
% Actors:            & The user \\ \hline
% Preconditions:     &  \\ \hline
% Post-conditions:   & \begin{itemize}
%     \item A diagram is detected.
%     \item A new temporary file is created.
% \end{itemize} \\ \hline
% Normal flows:      & \begin{enumerate}
%     \item The user opens the app.
%     \item The user chooses to create a blank diagram.
%     \item The user input diagram name.
%     \item System show file browser with default saving location.
%     \item The user chooses location to save diagram.
%     \item System create new file in storage and go to modify page.
% \end{enumerate} \\ \hline
% Alternative flows: & \begin{itemize}
%     \item {2a The user choose to scan a new diagram directly.}
%     \begin{itemize}
%         \item 2a1 Use case continue with table \ref{Tab:UC-3.1}.
%         \item 2a2 Use case continue with step 3.
%     \end{itemize}
%     \item {2b The user choose to scan a new diagram from image.}
%     \begin{itemize}
%         \item 2a1 Use case continue with table \ref{Tab:UC-3.2}.
%         \item 2a2 Use case continue with step 3.
%     \end{itemize}
% \end{itemize} \\ \hline
% Exception: & \begin{itemize}
%     \item {3a The user do not input diagram name.}
%     \begin{itemize}
%         \item 3a1 System uses default name (New Diagram).
%         \item 3a2 Use case continue with step 4.
%     \end{itemize}
%     \item {3b The user delete default name and leave name blank.}
%     \begin{itemize}
%         \item 3b1 System shows an error alert and let user try again.
%         \item 3b2 Use case continue with step 3.
%     \end{itemize}
%     \item {5a if the app has not been granted access to storage.}
%     \begin{itemize}
%         \item 5a1 System shows access permission notification.
%         \item 5a2 The user allow storage permission.
%         \item 5a3 Use case continue with step 6.
%     \end{itemize}
%     \item {5b If system cannot create a new file.}
%     \begin{itemize}
%         \item 5b1 System show error and return to home screen.
%         \item 5b2 Use case stop
%     \end{itemize}
% \end{itemize} \\ \hline
% \end{tabular}
% \captionof{table}{\label{Tab:UC-3} Use case: Create new diagram.}
% \end{table}

% \begin{table}[]
% \begin{tabular}{| m{4cm} | m{11cm} |}
% \hline
% Use case ID:       & UC-3.1 \\ \hline
% Use Case Name:     & Scan diagram with camera. \\ \hline
% Triggering Event:  & The user wants to scan a new diagram with camera. \\ \hline
% Brief Description: & The user chooses to scan device camera in create options. \\ \hline
% Actors:            & The user \\ \hline
% Preconditions:     & \\ \hline
% Post-conditions:   & \begin{itemize}
%     \item A diagram is detected.
% \end{itemize} \\ \hline
% Normal flows:      & \begin{enumerate}
%     \item The user takes a picture of the diagram.
%     \item The user adjusts the boundary of the diagram.
%     \item System shows result.
%     \item The user chooses “Done”. 
%     \item System continue with previous use case.
% \end{enumerate} \\ \hline
% Alternative flows: & \begin{itemize}
%     \item {2a The user want to retake the picture.}
%     \begin{itemize}
%         \item Use case continue with step 1.
%     \end{itemize}
% \end{itemize} \\ \hline
% Exception: & \begin{itemize}
%     \item {3a If there is no diagram detected.}
%     \begin{itemize}
%         \item 3a1 System show an error.
%         \item 3a2 The user choose to retake the picture.
%         \item 3a3 Use case continue with step 1.
%     \end{itemize}
% \end{itemize} \\ \hline
% \end{tabular}
% \captionof{table}{\label{Tab:UC-3.1} Use case: Scan diagram with camera.}
% \end{table}

% \begin{table}[]
% \begin{tabular}{| m{4cm} | m{11cm} |}
% \hline
% Use case ID:       & UC-3.2 \\ \hline
% Use Case Name:     & Scan from image. \\ \hline
% Triggering Event:  & The user clicks on “Scan from image” button in create options. \\ \hline
% Brief Description: & The user chooses to scan a new diagram from an image.\\ \hline
% Actors:            & The user \\ \hline
% Preconditions:     & N/A\\ \hline
% Post-conditions:   & \begin{itemize}
%     \item A diagram is detected.
% \end{itemize} \\ \hline
% Normal flows:      & \begin{enumerate}
%     \item System shows the storage browser.
%     \item The user chooses a picture.
%     \item System shows the full picture.
%     \item The user adjusts the boundary of the diagram.
%     \item The user chooses “Done”. 
%     \item System save the result and continue with previous use case.
% \end{enumerate} \\ \hline
% Alternative flows: & N/A\\ \hline
% Exception: & \begin{itemize}
%     \item {2a The user chooses a file that is not a picture.}
%     \begin{itemize}
%         \item 2a1 System show a warning
%         \item 2a2 The user choose to cancel.
%         \item 2a3 Use case stop.
%     \end{itemize}
%     \item {3a there is no diagram detected.}
%     \begin{itemize}
%         \item 3a1 System show an error.
%         \item 3a2 The user choose to retake the picture.
%         \item 3a3 Use case continue with step 1.
%     \end{itemize}
% \end{itemize} \\ \hline
% \end{tabular}
% \captionof{table}{\label{Tab:UC-3.2} Use case: Scan from image.}
% \end{table}

% \begin{table}[]
% \begin{tabular}{| m{4cm} | m{11cm} |}
% \hline
% Use case ID:       & UC-4 \\ \hline
% Use Case Name:     & Import file. \\ \hline
% Triggering Event:  & The user clicks on “Import file” button. \\ \hline
% Brief Description: & The user is in home page and want to import a file in storage to the app. \\ \hline
% Actors:            & The user \\ \hline
% Preconditions:     & N/A\\ \hline
% Post-conditions:   & \begin{itemize}
%     \item A new diagram is imported and show in home screen.
% \end{itemize} \\ \hline
% Normal flows:      & \begin{enumerate}
%     \item The user chooses to import from local storage.
%     \item System shows the storage browser.
%     \item The user chooses a file.
%     \item System save file location and show new diagram in home screen.
% \end{enumerate} \\ \hline
% Alternative flows: & \begin{itemize}
%     \item {2a The user choose to import from Google drive.}
%     \begin{itemize}
%         \item 2a1 The user login to Google Drive.
%         \item 2a2 System shows Google Drive file browser.
%         \item 2a3 Use case continue with step 3.
%     \end{itemize}
% \end{itemize} \\ \hline
% Exception: & \begin{itemize}
%     \item {3a The user chooses a file that is compatible or not accessible.}
%     \begin{itemize}
%         \item 3a1 System show a warning.
%         \item 3a2 The user choose to cancel.
%         \item 3a3 Use case stop.
%     \end{itemize}
% \end{itemize} \\ \hline
% \end{tabular}
% \captionof{table}{\label{Tab:UC-4} Use case: Import file.}
% \end{table}

% \begin{table}[]
% \begin{tabular}{| m{4cm} | m{11cm} |}
% \hline
% Use case ID:       & UC-5 \\ \hline
% Use Case Name:     & Export file. \\ \hline
% Triggering Event:  & The user clicks on “Export” button in diagram options. \\ \hline
% Brief Description: & The user wants to export a diagram. \\ \hline
% Actors:            & The user \\ \hline
% Preconditions:     & N/A\\ \hline
% Post-conditions:   & \begin{itemize}
%     \item A new diagram file is created.
% \end{itemize} \\ \hline
% Normal flows:      & \begin{enumerate}
%     \item The user selects a diagram and choose “Export”.
%     \item The user selects “Export as an image”.
%     \item The user chooses save location and select “Done”.
%     \item System create an image of selected diagram.
% \end{enumerate} \\ \hline
% Alternative flows: & \begin{itemize}
%     \item {2a The user select “Export as .pdf file”.}
%     \begin{itemize}
%         \item 2a1 The user chooses save location and select “Done”.
%         \item 2a2 System create an pdf file of selected diagram.
%         \item 2a3 Use case stop.
%     \end{itemize}
%     \item {2b The user select “Export as .drawio file”.}
%     \begin{itemize}
%         \item 2b1 The user chooses save location and select “Done”.
%         \item 2b2 System create a drawio file of selected diagram.
%         \item 2b3 Use case stop.
%     \end{itemize}
% \end{itemize} \\ \hline
% Exception: & \begin{itemize}
%     \item {3a If the diagram is blank, show a warning and allow user to cancel.}
%     \begin{itemize}
%         \item 3a1 The user cancel the operation.
%         \item Use case stop.
%     \end{itemize}
%     \item {3b if the app has not been granted access to storage.}
%     \begin{itemize}
%         \item 3b1 System shows access permission notification.
%         \item 3b2 The user allow storage permission.
%         \item 3b3 Use case continue with step 4.
%     \end{itemize}
% \end{itemize} \\ \hline
% \end{tabular}
% \captionof{table}{\label{Tab:UC-5} Use case: Export file.}
% \end{table}

% \begin{table}[]
% \begin{tabular}{| m{4cm} | m{11cm} |}
% \hline
% Use case ID:       & UC-6 \\ \hline
% Use Case Name:     & Modify diagram. \\ \hline
% Triggering Event:  & The user clicks on “Modify” button in diagram options. \\ \hline
% Brief Description: & The user wants to modify a diagram showed in home screen. \\ \hline
% Actors:            & The user \\ \hline
% Preconditions:     & N/A\\ \hline
% Post-conditions:   & \begin{itemize}
%     \item All changes is recorded and uploaded to server.
% \end{itemize} \\ \hline
% Normal flows:      & \begin{enumerate}
%     \item The user selects a diagram.
%     \item System shows diagram options.
%     \item The user selects “Modify”.
%     \item System go to modify page.
%     \item The user performs changes to the diagram.
%     \item The user selects “Save”.
%     \item System saves all changes and go back to home screen
% \end{enumerate} \\ \hline
% Alternative flows: & \begin{itemize}
%     \item {6a The user select ''Discard''.}
%     \begin{itemize}
%         \item 6a1 Systems show confirm message.
%         \item 6a2 The user select ''Yes''.
%         \item 6a3 Systems delete all changes and return to home screen.
%         \item 6a4 Use case stop.
%     \end{itemize}
% \end{itemize} \\ \hline
% Exception: & \begin{itemize}
%     \item {7a System cannot save file.}
%     \begin{itemize}
%         \item 7a1 System show saving error.
%         \item 7a2 The user selects cancel.
%         \item 7a3 Systems delete all changes and return to home screen.
%         \item 7a4 Use case stop.
%     \end{itemize}
%     \item {7b if the app has not been granted access to storage.}
%     \begin{itemize}
%         \item 7b1 System shows access permission notification.
%         \item 7b2 The user allow storage permission.
%         \item 7b3 Use case continue with step 4.

%     \end{itemize}
% \end{itemize} \\ \hline
% \end{tabular}
% \captionof{table}{\label{Tab:UC-6} Use case: Modify diagram.}
% \end{table}

% \begin{table}[]
% \begin{tabular}{| m{4cm} | m{11cm} |}
% \hline
% Use case ID:       & UC-7 \\ \hline
% Use Case Name:     & Delete Diagram. \\ \hline
% Triggering Event:  & The user choose ''Delete'' in diagram options. \\ \hline
% Brief Description: & The user want to delete a diagram from home screen. \\ \hline
% Actors:            & The user \\ \hline
% Preconditions:     & \begin{itemize}
%     \item The user is in home screen.
%     \item The user is the owner of the diagram.
%     \item The diagram is no longer shared with any other user.
% \end{itemize} \\ \hline
% Post-conditions:   & \begin{itemize}
%     \item The diagram is deleted in the database.
% \end{itemize} \\ \hline
% Normal flows:      & \begin{enumerate}
%     \item The user select a diagram, chooses ''Delete'' option.
%     \item System shows a warning message.
%     \item The user confirms to delete.
%     \item System deletes the diagram and shows the result message.
% \end{enumerate} \\ \hline
% Alternative flows: & \begin{itemize}
%     \item {4a The diagram is still shared to other users.}
%     \begin{itemize}
%         \item 4a1 System show error message and provide ''provoke all permission'' option.
%         \item 4a2 The user confirm.
%         \item 4a3 System delete all of share permission.
%         \item 4a4 Use case continue with step 4.
%     \end{itemize}
% \end{itemize} \\ \hline
% Exception: & \begin{itemize}
%     \item {4a The user is not the owner of the diagram.}
%     \begin{itemize}
%         \item 4a1 System show error message.
%         \item 4a2 Use case stops.
%     \end{itemize}
% \end{itemize} \\ \hline
% \end{tabular}
% \captionof{table}{\label{Tab:UC-7} Use case: Delete Diagram.}
% \end{table}

% \begin{table}[]
% \begin{tabular}{| m{4cm} | m{11cm} |}
% \hline
% Use case ID:       & UC-8 \\ \hline
% Use Case Name:     & Share. \\ \hline
% Scenario:          & . \\ \hline
% Triggering Event:  & . \\ \hline
% Brief Description: & . \\ \hline
% Actors:            & The user \\ \hline
% Preconditions:     & \\ \hline
% Post-conditions:   & \begin{itemize}
%     \item .
% \end{itemize} \\ \hline
% Normal flows:      & \begin{enumerate}
%     \item 
% \end{enumerate} \\ \hline
% Alternative flows: & \begin{itemize}
%     \item {.}
%     \begin{itemize}
%         \item 
%     \end{itemize}
% \end{itemize} \\ \hline
% Exception: & \begin{itemize}
%     \item {.}
%     \begin{itemize}
%         \item 
%     \end{itemize}
% \end{itemize} \\ \hline
% \end{tabular}
% \captionof{table}{\label{Tab:UC-8} Share.}
% \end{table}

% \begin{table}[]
% \begin{tabular}{| m{4cm} | m{11cm} |}
% \hline
% Use case ID:       & UC-9 \\ \hline
% Use Case Name:     & Set permission. \\ \hline
% Scenario:          & . \\ \hline
% Triggering Event:  & . \\ \hline
% Brief Description: & . \\ \hline
% Actors:            & The user \\ \hline
% Preconditions:     & \\ \hline
% Post-conditions:   & \begin{itemize}
%     \item .
% \end{itemize} \\ \hline
% Normal flows:      & \begin{enumerate}
%     \item 
% \end{enumerate} \\ \hline
% Alternative flows: & \begin{itemize}
%     \item {.}
%     \begin{itemize}
%         \item 
%     \end{itemize}
% \end{itemize} \\ \hline
% Exception: & \begin{itemize}
%     \item {.}
%     \begin{itemize}
%         \item 
%     \end{itemize}
% \end{itemize} \\ \hline
% \end{tabular}
% \captionof{table}{\label{Tab:UC-9} Set permission.}
% \end{table}

% \begin{table}[]
% \begin{tabular}{| m{4cm} | m{11cm} |}
% \hline
% Use case ID:       & UC-10 \\ \hline
% Use Case Name:     & View change history. \\ \hline
% Scenario:          & . \\ \hline
% Triggering Event:  & . \\ \hline
% Brief Description: & . \\ \hline
% Actors:            & The user \\ \hline
% Preconditions:     & \\ \hline
% Post-conditions:   & \begin{itemize}
%     \item .
% \end{itemize} \\ \hline
% Normal flows:      & \begin{enumerate}
%     \item 
% \end{enumerate} \\ \hline
% Alternative flows: & \begin{itemize}
%     \item {.}
%     \begin{itemize}
%         \item 
%     \end{itemize}
% \end{itemize} \\ \hline
% Exception: & \begin{itemize}
%     \item {.}
%     \begin{itemize}
%         \item 
%     \end{itemize}
% \end{itemize} \\ \hline
% \end{tabular}
% \captionof{table}{\label{Tab:UC-10} View change history.}
% \end{table}

% \begin{table}[]
% \begin{tabular}{| m{4cm} | m{11cm} |}
% \hline
% Use case ID:       & UC-11 \\ \hline
% Use Case Name:     & Revert.. \\ \hline
% Triggering Event:  & User wants to revert back to the previous version of the diagram. \\ \hline
% Brief Description: & User choose ''Revert'' on the version in the history change. \\ \hline
% Actors:            & The user \\ \hline
% Preconditions:     & \begin{itemize}
%     \item User is the owner or has ''Can edit'' permission.
%     \item The selected version is not the current one.
% \end{itemize} \\ \hline
% Post-conditions:   & \begin{itemize}
%     \item A reverted upload is sent.
% \end{itemize} \\ \hline
% Normal flows:      & \begin{enumerate}
%     \item 
% \end{enumerate} \\ \hline
% Alternative flows: & \begin{itemize}
%     \item {.}
%     \begin{itemize}
%         \item 
%     \end{itemize}
% \end{itemize} \\ \hline
% Exception: & \begin{itemize}
%     \item {.}
%     \begin{itemize}
%         \item 
%     \end{itemize}
% \end{itemize} \\ \hline
% \end{tabular}
% \captionof{table}{\label{Tab:UC-11} Revert..}
% \end{table}

% \begin{table}[]
% \begin{tabular}{| m{4cm} | m{11cm} |}
% \hline
% Use case ID:       & UC-12 \\ \hline
% Use Case Name:     & Logout. \\ \hline
% Triggering Event:  & The user is done using the app and want to logout. \\ \hline
% Brief Description: & The user chooses ''Log out'' and ends their logging session. \\ \hline
% Actors:            & The user \\ \hline
% Preconditions:     & \begin{itemize}
%     \item The user is logged in.
% \end{itemize} \\ \hline
% Post-conditions:   & \begin{itemize}
%     \item The user is logged out.
% \end{itemize} \\ \hline
% Normal flows:      & \begin{enumerate}
%     \item The user choose ''Option'' and ''Logout''.
%     \item System logs user out and invalidates the API key.
%     \item System redirects to login screen.
% \end{enumerate} \\ \hline
% Alternative flows: & N/A \\ \hline
% Exception: & N/A \\ \hline
% \end{tabular}
% \captionof{table}{\label{Tab:UC-12} Use case: Logout.}
% \end{table}
\chapter{Related works}\label{chap: Related works}
Our topic is not entirely new to the Blockchain community.
There tons of paper proposed the same problem as ours.
We analyzed the paper Towards on Blockchain Data Privacy Protection (Zexu Wang, Bin Wen and Ziqiang Luo) and find a very interesting approach to the wallet technical architecture. 
The paper focused on wallet keys protocols, like tree structure of the HD wallet, rules of deriving child keys from given parent keys, system privacy and how to establish a secure collaboration in blockchain network.
We may or may not implement their methods in our thesis since we haven't examine the impact of attacks on their complete system architecture. There are no number prove that their proposed system are perfectly secure.
In practice, BIP32 of Bitcoin wallet is widely used but the methods can not apply to Edwards curve. 
This is a great example of constructing a parent-child keys derivation. We will mention more of BIP32 on our final thesis report. 

\chapter{Conclusion and Future Work} \label{chap:conclusion}

% \section{conclusion}
% The field diagram detection is an interesting and complex study domain. Although there are many research about on-line diagram, there is still less research about recognizing off-line diagram. With this thesis proposal, we have achieved new background knowledge about the computer vision. We are able to access each procedures in image processing which is also a new field for us. We finds many approaches for handwriting flowchart recognition and we select the suitable approaches using Faster R-CNN, YOLO v4 and RefineDet.\\
% During the research process, our group still had many problems in teamwork, lack of time management skills which affects negatively our result. This problems may become a wall in our work in the future, which we need to overcome and improve our working process.

% \section{Challenges}
% \begin{itemize}
%     \item One of the biggest problem with server-client system is scaling or the amount of customers serving at the same time. Most of our experiment is conducted in the local machine so that there is a need to find a way to provide more stable service in the final product.
%     \item The target is to support a wide range of products so as many devices can install this app as possible. Then it can lead to performance issues and the need to find a balance between stable functioning and feature variety.
%     \item Security is also important since many of these data can be very crucial. All of the information sent or receive by the app and the server need to be secured in order to prevent data leak.
% \end{itemize}

% \section{Future work}
% In the future, we are going to implement the flowchart recognition system and also evaluate the capabilities of the programming languages and the support framework to reduce the constraints of our proposed system while solving all of the listed challenges.

\renewcommand{\bibname}{References}
\bibliography{refs}
\bibliographystyle{unsrt}


\end{document}
